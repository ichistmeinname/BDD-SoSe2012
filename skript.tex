\documentclass[ngerman]{scrartcl}
\KOMAoptions{paper=a4}
\KOMAoptions{fontsize=11pt}
\KOMAoptions{pagesize=pdftex}


%% Spracheinstellungen laden
\usepackage[ngerman]{babel}

%% Schriftart in der Ausgabe/Eingabe
\usepackage{geometry}
\usepackage[T1]{fontenc}
\usepackage{textcomp}
\usepackage[latin1]{inputenc}
\usepackage{enumitem}
\usepackage{amssymb}
\usepackage{amsmath}
\usepackage{amsthm}
\usepackage{listings}
\usepackage{ulem}
\usepackage{dsfont}
\usepackage{empheq}
\usepackage[colorlinks=true]{hyperref}
\setlength{\parindent}{0em}
\geometry{a4paper,top=20mm,bottom=25mm,left=20mm,right=20mm,footskip=10mm}

\pagestyle{myheadings}
\markright{Binary Decision Diagrams - SoSe 12}

\swapnumbers
\newtheoremstyle{custom}{10pt}{10pt}{\normalsize}{ }{\bfseries}{
}{\newline}{ }
\theoremstyle{custom}

\newtheorem{mdef}{Definition} \numberwithin{mdef}{subsection}
\newtheorem{mex}[mdef]{Beispiel}
\newtheorem{ms}[mdef]{Satz}
\newtheorem{ml}[mdef]{Lemma}
\newtheorem{mbez}[mdef]{Bezeichnungen}
\newtheorem{mr}[mdef]{Regel}
\newtheorem*{ex}{Beispiel}
\newtheorem*{bw}{Beweis}
\newtheorem{mk}[mdef]{Konvention}
\newtheorem{obdd}[mdef]{OBDDs als Zeigergeflecht}
\newtheorem{algred}[mdef]{Algorithmus Reduce}
\newtheorem{algval}[mdef]{Auswertungsalgorithmus Value}
\newtheorem{algsat}[mdef]{Erf\"ullbarkeitsalgorithmus Satisfy}
\newtheorem{algequ}[mdef]{Algorithmus Equiv}
\newtheorem{algneg}[mdef]{Algorithmus Negation}
\newtheorem{algrem}[mdef]{Algorithmus Remove}
\newtheorem{algdec}[mdef]{Algorithmus Decisive}
\newtheorem{binsys}[mdef]{Bin\"are Synthese bei QOBDDs}
\newtheorem{apply-robdd}[mdef]{Algorithmus Apply bei ROBDDs}

\newcommand{\qd}{\begin{flushright}$\square$\end{flushright}}
\newcommand{\ax}{\mathcal{A}(X)}
\newcommand{\val}{\underline{val}_v}
\newcommand{\0}{\mathbf{0}}
\newcommand{\1}{\mathbf{L}}
\newcommand{\at}{\mathcal{A}t}
\newcommand{\ba}{$(V, \sqcap, \sqcup, \bar~, \0, \1)~$}
\newcommand{\bel}{$V: X \rightarrow \mathds{B}$}
\newcommand{\bol}{\mathds{B}^n \rightarrow \mathds{B}}
\newcommand{\bolf}{$f: \bol~$}
\newcommand{\then}{\underline{then}}
\newcommand{\el}{\underline{else}}
\newcommand{\var}{\underline{var}}
\newcommand{\eins}{\underline{eins}}
\newcommand{\set}{\underline{set}}
\newcommand{\cons}{\underline{cons}}
\newcommand{\sg}{$(X,\mathcal{W})~$}
\newcommand{\gms}{$\lbrack Q; w_1, \dots w_n \rbrack$}
\newcommand{\W}{\mathcal{W}}
\newcommand{\ith}{\underline{ith}}
\newcommand{\ww}{\preccurlyeq _{I}}

\begin{document}
\begin{titlepage}
\thispagestyle{empty}
\begin{small}
\vfill {Christian-Albrechts-Universit\"at zu Kiel\\ 
Institut f\"ur Informatik \\ 
Dozent: Prof. Berghammer\\
Sommersemester 2012}
\end{small}

\vspace{5cm}
\begin{center}
\begin{Large}
 {\textsf{\textbf{BDD - Binary Decision Diagrams
}}}
\end{Large}

\vspace{1cm}
Mitschrift von Sandra Dylus
\end{center}
\begin{small}
\begin{center}
Letzte Aktualisierung: \today
\end{center}
\end{small}
\vfill

\begin{small} 
\textbf{Kontaktadresse}: sad@informatik.uni-kiel.de\\
\textbf{Hinweis}: Keine Garantie auf Richtigkeit, bei Fehlerfindung bitte kontaktieren.
\end{small}

\end{titlepage}
\tableofcontents
\newpage

\section{Grundlagen}
\subsection{Aussagenlogik}
Was ist eine Aussage? - Ein sprachliches Gebilde, von dem es
Sinn macht, zu sagen, es sei wahr oder falsch.

\begin{ex} Kiel liegt an der Ostsee. Kiel liegt an der Nordsee.\end{ex}
\begin{ex} $x \in A$, $X=Y$, $\exists x: A(x)$, $\forall x :
  B(x)$\end{ex}

Kurze Wiederholung von Junktoren:
\begin{enumerate}
\item Negation: $\neg A$
\item Konjunktion: $A \wedge B$
\item Disjunktion: $A \vee B$
\item Implikation: $A \rightarrow B$
\item \"Aquivalenz: $A \leftrightarrow B$
\end{enumerate}

\begin{mdef} Sei $X$ eine Menge von Aussagenvariablen . Dann ist die
  Menge $\mathcal{A}(X)$ der \underline{aussagenlogischen Formeln}
  \"uber $X$ wie folgt definiert:
\begin{enumerate}
\item F\"ur alle $a \in X$ gilt $a \in \mathcal{A}(X)$
\item F\"ur alle $\phi \in \mathcal{A}(X)$ gilt $(\neg \phi) \in
  \mathcal{A}(X)$
\item F\"ur alle $\phi, \psi \in \mathcal{A}(X)$ gelten
  \begin{itemize}
   \item $(\phi \wedge \psi) \in \mathcal{A}(X)$
   \item $(\phi \vee \psi) \in \mathcal{A}(X)$
   \item $(\phi \rightarrow \psi) \in \mathcal{A}(X)$
   \item $(\phi \leftrightarrow \psi) \in \mathcal{A}(X)$
    \end{itemize}
\item Es gibt keine Elemente in $\mathcal{A}(X)$, au\ss er denen, die
  (1.) bis (3.) zulassen.
\end{enumerate}
\qed
\end{mdef}

Die Negation ($\neg$) bindet am st\"arksten, danach die Konjunktion
($\wedge$) und Disjunktion ($\vee$) und zuletzt die Implikation
($\rightarrow$) sowie die \"Aquivalenz ($\leftrightarrow$) (Vorrangsregeln).

\begin{ex}
$X = \{a,b,c,d,e\}$\\
$(\underbrace{(\underbrace{(\underbrace{\neg a}) \vee (\underbrace{\neg b})}) \vee
(\underbrace{(\underbrace{c \wedge d}) \wedge e})})$ \\
$\huge\rightsquigarrow (\neg a \vee \neg b) \vee ((c
\wedge d) \wedge e)$ \\
$\rightsquigarrow \neg a \vee \neg b \vee (c
\wedge d \wedge e)$ \\
\end{ex}

\begin{mdef}Es ist $\mathds{B}:=\{\mathbf{0}, \mathbf{1}\}$ die Menge der
  Wahrheitswerte.\qed 
\end{mdef}
 
\begin{mdef} Eine \emph{Belegung} ist eine Funktion $v: X \rightarrow
  \mathds{B}$. Zu $a \in X$ hei\ss t $v(a)$ die \emph{Belegung} von
  $a$. \qed
\end{mdef}

\begin{mdef} Die Funktionen $\bar{ }: \mathds{B} \rightarrow
  \mathds{B}, +, \cdot: \mathds{B} \rightarrow \mathds{B}$ sind durch
  die folgende Tafeln festgelegt:\\
\begin{tabular}{c|cc}
$a$ & $0$ & $1$ \\
\hline $\overline{a}$ & $1$ & $0$
\end{tabular}
\begin{tabular}{c|cccc}
+ & & & & \\
\hline $a$ & $0$ & $0$ & $1$ & $1$ \\
$b$ & $0$ & $1$ & $0$ & $1$\\
$a+b$ & $0$ & $1$ & $1$ & $1$ \\
\end{tabular}
\begin{tabular}{c|cccc}
$\cdot$ & & & & \\
\hline $a$ & $0$ & $0$ & $1$ & $1$ \\
$b$ & $0$ & $1$ & $0$ & $1$\\
$a \cdot b$ & $0$ & $0$ & $0$ & $1$ \\
\end{tabular}
\end{mdef}

\begin{mdef} Zu einer Belegung $v: X \rightarrow \mathds{B}$ ist der
  Wert $\underline{val}_v(\phi)$ f\"ur $\phi \in \ax$ induktiv wie
  folgt festgelegt:
\begin{enumerate}
\item $\val(a) = v(a)$ f\"ur $a \in X$
\item $\val(\neg \phi) = \overline{\val(\phi)}$ f\"ur $\phi \in \ax$
\item $\val(\phi \vee \psi) = \val(\phi) + \val(\psi)$ f\"ur alle $\phi,
  \psi \in \ax$
\item $\val(\phi \wedge \psi) = \val(\phi) \cdot \val(\psi)$
  f\"ur alle $\phi, \psi \in \ax$
\item $\val(\phi \rightarrow \psi) = \overline{\val(\phi)} +
  \val(\psi)$ f\"ur alle $\phi, \psi \in \ax$
\item $\val(\phi \leftrightarrow \psi) = \val(\phi \rightarrow \psi) \cdot
  \val(\psi \rightarrow \phi)$ f\"ur alle $\phi, \psi \in \ax$
\end{enumerate}
\qed
\end{mdef}

\begin{mdef}
Die Relation $\Leftrightarrow ~ \subseteq \ax \times \ax$ ist f\"ur alle
$\phi, \psi \in \ax$ definiert durch:\\
\begin{equation*}
\phi \Leftrightarrow \psi :\Longleftrightarrow \text{ f\"ur alle } v \in
\mathds{B}^X \text{ gilt } \val(\phi) = \val(\psi)
\end{equation*}
$\phi$ und $\psi$ sind \emph{logisch \"aquivalent}, falls $\phi
\Leftrightarrow \psi$ gilt. \qed
\end{mdef}

\begin{mdef}
Die Relation $\Rightarrow ~ \subseteq \ax \times \ax$ ist f\"ur alle
$\phi, \psi \in \ax$ definiert durch:\\
\begin{equation*}
\phi \Rightarrow \psi :\Longleftrightarrow \text{ f\"ur alle } v \in
\mathds{B}^X \text{ gilt, wenn } \val{\phi}=1 \text{, dann }
\val{\psi}=1
\end{equation*}
$\phi$ \emph{impliziert logisch} $\psi$, falls $\phi
\Rightarrow \psi$ gilt. \qed
\end{mdef}

\begin{ms} F\"ur alle $\phi, \psi \in \ax$ gelten:
\begin{enumerate} 
\item $\phi \Leftrightarrow \psi$ ist \"aquivalent zu $\val(\phi
  \leftrightarrow \psi)=1$ f\"ur alle $v \in \mathds{B}^X$
\item $\phi \Rightarrow \psi$ ist \"aquivalent zu $\val(\phi
  \rightarrow \psi) = 1$ f\"ur alle $v \in \mathds{B}^X$
\end{enumerate}
\qed
\end{ms}

\begin{ms}
\begin{enumerate}
\item $\Leftrightarrow$ ist \"Aquivalenzrelation auf $\ax$,
  d.h. reflexiv, symmetrisch und transitiv
\item $\Rightarrow$ ist Quasiordnung auf $\ax$, d.h. reflexiv und
  transitiv
\item f\"ur alle $\phi, \psi, \rho \in \ax$ gilt: $\phi \Rightarrow
  \psi$ und $\psi \Rightarrow \rho$ impliziert $\phi \Rightarrow \rho$
\item f\"ur alle $\phi, \psi \in \ax$ gilt $\phi \Leftrightarrow \psi$
  genau dann, wenn $\phi \Rightarrow \psi$ und $\psi \Rightarrow \phi$
  gelten
\end{enumerate}
\qed
\end{ms}

\begin{mex}
Seien $a,b,c \in X$. Dann gilt:
\begin{eqnarray*}
a \rightarrow (b \rightarrow c) &\stackrel{Def.}{\Leftrightarrow}& \neg a \vee (b
\rightarrow c) \\
&\stackrel{Def.}{\Leftrightarrow}& \neg a \vee (\neg b \vee c)\\
&\stackrel{Ass.}{\Leftrightarrow}& (\neg a \vee \neg b) \vee c \\
&\stackrel{De M.}{\Leftrightarrow}& \neg (a \wedge b) \vee c \\
&\stackrel{Def.}{\Leftrightarrow}& (a \wedge b) \rightarrow c
\end{eqnarray*}
\qed
\end{mex}

\subsection{Boolesche Algebra}
\begin{mdef} Ein \emph{Verband} ist eine algebraische Struktrur $(V,
  \sqcap, \sqcup)$ mit $V \not = \emptyset$ und $\sqcap, \sqcup: V \times
  V \rightarrow V$, so dass f\"ur alle $x, y \in V$ gilt:
\begin{enumerate}
\item $x \sqcup (y \sqcup z) = (x \sqcup y) \sqcup z$
\item $x \sqcap (y \sqcap z) = (x \sqcap y) \sqcap z$
\item $x \sqcup y = y \sqcup x$
\item $x \sqcap y = y \sqcap x$
\item $x \sqcap (y \sqcup x) = x$
\item $x \sqcup (y \sqcap x) = x$
\end{enumerate}
\qed
\end{mdef}

\begin{mex}

\begin{enumerate}

\item $(\mathds{B}, \cdot, +)$ ist Verband
\item $(2^M, \cap, \cup)$ ist Verband
\item $(\mathds{N}, \underline{ggT}, \underline{kgV})$ ist Verband
\item $(\mathds{N}, \underline{min}, \underline{max})$ ist Verband
\end{enumerate}
\qed
\end{mex}

\begin{ms}
Es sei $(V, \sqcap, \sqcup)$ ein Verband. Dann gelten f\"ur alle $x,y
\in V$ die folgenden Eigenschaften:
\begin{enumerate}
\item $x \sqcup x = x \text{ und } x \sqcap x = x$
\item $x \sqcap y = x \Leftrightarrow x \sqcup y = y$
\end{enumerate}
\end{ms}

\begin{bw}~\\
(1)
\begin{eqnarray*}
x \sqcup x &=& x \sqcup (x \sqcap (x \sqcup x))\\
&=& x \sqcup ((x \sqcup x) \sqcap x)\\
&=& x
\end{eqnarray*}
\begin{eqnarray*}
x \sqcap x &=& x \sqcap (x \sqcup (x \sqcap x)) \\
&=& x \sqcap ((x \sqcap x) \sqcup x) \\
&=& x
\end{eqnarray*} 
(2) 
$\Rightarrow$ Sei $x \sqcap y = x$, dann
\begin{eqnarray*}
x \sqcup y &=& (x \sqcap y) \sqcup y \\
&=& y \sqcup (x \sqcap y) \\
&=& y
\end{eqnarray*}
$\Leftarrow$ Sei $x \sqcup y = y$, dann
\begin{eqnarray*}
x \sqcap y &=& x \sqcap (x \sqcup y) \\
&=& x \sqcap (y \sqcup x) \\
&=& x
\end{eqnarray*}
\qed
\end{bw}

\begin{ms} Es sei $(V, \sqcap, \sqcup)$ ein Verband. Definiert man auf
  $V$ eine Relation $\sqsubseteq ~\subseteq V \times V$ f\"ur alle
  $x,y \in V$ durch $x \sqsubseteq y :\Longleftrightarrow x \sqcap y =
  x (\Longleftrightarrow x \sqcup y = y$, so ist $(V, \sqsubseteq)$
  eine geordnete Menge.
\end{ms}

\begin{bw}
Reflexivit\"at: Sei $x \in V$
\begin{equation*}
x \sqsubseteq x \stackrel{Def.}{\Leftrightarrow} x \sqcap x = x
\stackrel{Satz 1.2.3 (1)}{\Leftrightarrow}
\underline{wahr}
\end{equation*}
Antisymmetrie: Seien $x, y \in V$
\begin{equation*}
x \sqsubseteq y \wedge y \sqsubseteq x \stackrel{Def.}{\Leftrightarrow} x \sqcap y = x
\wedge y \sqcap x = y \stackrel{Komm.}{\Leftrightarrow} x= y
\end{equation*}
Transitivit\"at: Seien $x,y, z \in V$
\begin{eqnarray*}
x \sqsubseteq y \wedge y \sqsubseteq z
&\stackrel{Def.}{\Leftrightarrow}& x \sqcap y = x \wedge y \sqcap z =
y \\
&\stackrel{Vor. + Ass.}{\Rightarrow}& x \sqcap z = (x \sqcap y) \sqcap z
= x \sqcap (y \sqcap z) = x \sqcap y = x \\
&\stackrel{Def.}{\Leftrightarrow}& x \sqsubseteq z
\end{eqnarray*}
\qed
\end{bw}

\begin{ex}
\begin{enumerate}
\item $(V, \sqcap, \sqcup, \sqsubseteq)$
\item $(\mathds{B}, \cdot, +, \Rightarrow)$
\item $(2^M, \cap, \cup, \subseteq)$
\item $(\mathds{N}, \underline{ggT}, \underline{kgV}, \setminus)$
\item $(\mathds{N}, \underline{min}, \underline{max}, \le)$
\end{enumerate}
\end{ex}

\begin{mdef} Ein Verband $(V, \sqcap, \sqcup)$ hei\ss t distributiv,
  falls f\"ur alle $x, y, z \in V$ gelten:
\begin{enumerate}
\item $x \sqcup (y \sqcap z) = (x \sqcup y) \sqcap (x \sqcup z)$
\item $x \sqcap (y \sqcup z) = (x \sqcap y) \sqcup (x \sqcap z)$
\end{enumerate}
\qed
\end{mdef}

\begin{mdef}
Eine Boolesche Algebra ist eine algebraische Struktur $(V, \sqcap,
\sqcup, \bar~, \mathbf{0}, \mathbf{L})$ mit $\sqcap, \sqcup: V \times V
\rightarrow V$, $\bar~: V \rightarrow V$, $\mathbf{0}, \mathbf{L} \in
V$, so dass folgende Eigenschaften gelten:
\begin{enumerate}
\item $(V, \sqcap, \sqcup)$ ist ein Verband
\item f\"ur alle $x \in V$ gilt $x \sqcap \neg x = \mathbf{0}$ und
  $x \sqcup \neg x = \mathbf{L}$
\end{enumerate}
\qed
\end{mdef}

Daraus kann man Folgendes observieren:
\begin{eqnarray*}
x \sqsubseteq \mathbf{L} &\Leftrightarrow& x \sqcup \mathbf{L} = \mathbf{L} \\
&\Leftrightarrow& x \sqcup (x \sqcup \neg x) = x \sqcup \neg x\\
&\Leftrightarrow& (x \sqcup x \sqcup) \neg x = x \sqcup \neg x\\
&\Leftrightarrow& x \sqcup \neg x = x \sqcup \neg x
\end{eqnarray*}
\begin{eqnarray*}
x \sqsubseteq \mathbf{0} &\Leftrightarrow& \mathbf{0} \sqcap x =
\mathbf{0} \\
&\Leftrightarrow& (x \sqcap \neg x) \sqcap x = x \sqcap \neg x \\
&\Leftrightarrow& x \sqcap \neg x = x \sqcap \neg x
\end{eqnarray*}
Also ist $\mathbf{L}$ gr\"o\ss tes Element und $\mathbf{0}$ kleinstes Element von $V$ in $(V,
\sqsubseteq)$.

\begin{table}[h]
\center
\begin{tabular}{c|c|c}
& Boolesche Algebra & Aussagenlogik \\
\hline Tr\"agermenge & $V$ & $\ax$ \\
Operationen & $\sqcap$ & $\wedge$ \\
& $\sqcup$ & $\vee$ \\
& $\bar ~$ & $\neg$ \\
Elemente & $\mathbf{O}$ & \underline{falsch} \\
& $\mathbf{L}$& \underline{wahr}\\
Ordnung & $\sqsubseteq$ & $\rightarrow$ \\
Gleichheit & $=$ & $\leftrightarrow$
\end{tabular}
\caption{Vergleich der Booleschen Algebra und Aussagenlogik}
\end{table}

\begin{ms} \label{s127}
Sei $(V, \sqcap, \sqcup, \bar ~, \mathbf{O}, \mathbf{L})$ eine
Boolesche Algebra. Dann gelten f\"ur alle $x,y \in V$ folgende
Eigenschaften:
\begin{enumerate}
\item $\overline{\overline{x}} = x$
\item $\overline{x \sqcap y} = \overline{x} \sqcup \overline{y}$ und
  $\overline{x \sqcup y} = \overline{x} \sqcap \overline{y}$
\item $x \sqsubseteq y \Leftrightarrow \neg x \sqcap y = \mathbf{L}
  \Leftrightarrow x \sqcap \neg y = \mathbf{0}$
\item $\overline{\mathbf{0}} = \mathbf{L}$ und $\overline{\mathbf{L}}
  = \mathbf{0}$
\end{enumerate}
\qed
\end{ms}

\begin{ms}
Gilt eine Gleichung $t_1 = t_2$ in allen Booleschen Verb\"anden, so
gilt auch $t_1^d = t_2^d$, wobei $t_i^d$ aus $t_i$ dadurch entsteht,
dass man $\sqcap$ und $\sqcup$ sowie $\0$ und $\1$ vertauscht.
\qed
\end{ms}

\begin{mdef}
Es sei $(V,\sqcap, \sqcup, \bar~, \0, \1)$ eine Boolesche Algebra,
dann hei\ss t $a \in V$ \emph{Atom}, falls $a \not= \0$ und
f\"ur alle $x \in V$ gilt $x \sqsubseteq a \Rightarrow x = \0 \vee x =
a$. $\at(V)$ sei die Menge der Atome von $V$.\\
\qed
\end{mdef}

Daraus folgern wir: falls $\vert V \vert < \infty$ und $\vert V \vert
\geq 2$, dann gilt $\at(V) \not= \0$ und f\"ur alle $x \in V$ gibt es
$a \in \at(V)$ mit $a \sqsubseteq x$.

\begin{ml}
In einer Booleschen Algebra \ba ~gelten die folgenden Eigenschaften:
\begin{enumerate}
\item f\"ur alle $a, b \in \at(V)$ mit $a \not= b$ gilt $a \sqcap b =
  \0$
\item f\"ur alle $a \in \at(V)$ mit $x \in V$ gilt $a \not \sqsubseteq
  x \Rightarrow a \sqsubseteq \overline{x}$
\item f\"ur alle $a \in \at(V)$ und $x_1, \dots, x_n \in V, n \geq 1$
  gilt $a \sqsubseteq \bigsqcup_{i=1}^{n} x_i \Leftrightarrow \exists
  j \in \{1, \dots, n\}: a \sqsubseteq x_j$
\end{enumerate}
\end{ml}

\begin{bw} ~\\
(1) Angenommen $a \sqcap b \not= \0$. Dann gilt $a \sqcap b = a$, da
$a \sqcap b \sqsubseteq a$, und $a \sqcap b = b$, da $a \sqcap b =
b$. Also gilt $a=b$ und das widerspricht $a \not= b$.\\
(2) $a \not \sqsubseteq x \Leftrightarrow a \sqcap \overline{x} \not =
a$.\\ Da $a \sqcap \overline{x} \sqsubseteq a$ folgt $a \sqcap
\overline{x} = \0$. Also $a \sqsubset \overline{x}$ nach \hyperref[s127]{Satz 1.2.7}
(3).\\
(3)
\begin{eqnarray*}
\not \exists j \in \{1, \dots, n\}: a \sqsubseteq x_j &\Leftrightarrow&
\forall j \in \{1, \dots, n\}: a \not \sqsubseteq x_j \\
&\Rightarrow& \forall j \in \{1, \dots, n\}: a \sqsubseteq
\overline{x_j} \\
&\Rightarrow& a \sqcap \bigsqcup_{i=1}^{n} x_i = \bigsqcup_{i=1}^{n}
(a \sqcap x_i) = \bigsqcup_{i=1}^{n} \0 \\
&\Rightarrow& a \not \sqsubseteq \bigsqcup_{i=1}^{n} x_i ~\text{ da }a
  \in \at(V)
\end{eqnarray*}
\end{bw}

\begin{ms}
Ist \ba eine endliche Boolesche Algebra, so ist die Funktion
\begin{equation*}
f: V \rightarrow 2^{\at(V)}, \hspace{2cm} f(x) = \lbrace a \in \at(V) \vert a \sqsubseteq
  x\rbrace
\end{equation*}
eine bijektive Funktion. Also gilt $\vert V \vert = 2^{\vert
    \at(V) \vert}$.
\end{ms}

\begin{bw} ~\\
\underline{Injektivit\"at:} Seien $x,y \in V$ mit $x \not= y$, z.z. ist $f(x) \not=
f(y)$. Aus $x \not= y$ folgt ($x \not \sqsubseteq y$ oder $y \not
\sqsubseteq x$). Es gelte o.B.d.A. $x \not \sqsubseteq y$. Dann gilt
$x \sqcap \overline{y} \not= \0$. Also gibt es, da $\vert V \vert <
\infty$, ein Atom $a \in \at(V)$ mit $a \sqsubseteq x \sqcap
\overline{y}$.\\

(1) $a \sqsubseteq x$ \hspace{3cm}
(2) $a \sqsubseteq \overline{y}$\\

Aus (1) und $a \in \at(V)$ folgt $a \in f(x)$. Aus (2) folgt $a \not
\sqsubseteq y$. W\"are $a \subseteq y$, dann golte $a \sqsubseteq
\overline{y}$ und $a \subseteq y$, also $a \sqsubseteq y \sqcap
\overline{y} = \0$, daraus folgt wiederum $a = \0$, dies steht aber im
Widerspruch zu $a \in \at(V)$. Also gilt $a \notin f(y)$.\\
Da $a \in f(x)$ und $a \notin f(y)$ gilt $f(x) \not= f(y)$.
\\
\underline{Surjektivit\"at:} Sei $A \in 2^{\at(V)}$. Da $\vert V \vert
< \infty$ gibt es $a_1, \cdots a_n \in \at(V)$ mit $A = \{a_1, \cdots,
a_n$.\\
Beh.: $f(\bigsqcup_{i=1}^{n} a_i) = \{a_1, \cdots, a_n\}$ f\"ur $n
\geq 1$ und $f(\0) = \emptyset$ f\"ur $n=0$\\
Bew.: $\subseteq$\\
Sei $b \in f (\bigsqcup_{i=1}^{n} a_i)$, d.h. $b \in \at(V)$ unf $b
\sqsubseteq \bigsqcup_{i=1}^{n} a_i$. Lemma 1.2.10 (3) zeigt, dass es
$j \in \{1, \cdots, n\}$ mit $b \sqsubseteq a_j$ gibt. Wegen $b \in
\at(V)$ gilt $b = a_j$, also $b \in A$.\\
$\supseteq$\\
Sei $b \in A$, d.h. $b = a_j$ f\"ur ein $j \in \{1, \dots, n\}$. Dann
gilt $b \in \at(V)$ und $b \sqsubseteq \bigsqcup_{i=1}^{n} a_i$ (da $b
\sqcup \bigsqcup_{i=1}^{n} a_i = a_j \sqcup \bigsqcup_{i=1}^{n} a_i =
\bigsqcup_{i=1}^{n} a_i$). Die Definition von $f$ bringt letztendlich
$b \in f(\bigsqcup_{i=1}^{n} a_i)$.\\
\qed
\end{bw}

Weiterhin gelten f\"ur $f$ noch folgene Eigenschaften:
\begin{enumerate}
\item $f(x \sqcup y) = f(x) \cup f(y)$
\item $f(x \sqcap y) = f(x) \cap (fy)$
\item $f(\overline{x}) = \overline{f(x)} = \at(V) \setminus f(x)$
\item $f(\0) = \emptyset$
\item $f(\1) = \at(V)$
\end{enumerate}

\subsection{Boolesche Funktionen}
\begin{mdef}
Ist \ba eine Boolesche Algebra und $n \in V$ mit $n \geq 1$, dann ist
$V_n$ die Menge der n-stelligen Funktionen $f: V^n \rightarrow V$.
\qed
\end{mdef}
\begin{mdef}
Zu $(\mathds{B}, \cdot, +, \bar~, \0, \1)$ und $n \geq 1$ hei\ss t $f
\in \mathds{B}_n$ \underline{Boolesche Funktion} (Schaltfunktion).
\qed
\end{mdef}

\begin{ms}
Ist \ba eine Boolesche Algebra, so wird auch $(V_n,
\widetilde{\sqcap}, \widetilde{\sqcup}, \widetilde{\bar~},
\widetilde{\0}, \widetilde{\1})$ zu einer Booleschen Algebra, indem
man definiert:\\
$\widetilde{\0}: V^n \rightarrow V \hspace{1cm} \widetilde{\0}(x_1,
\dots, x_n) = \0$\\
$\widetilde{\1}: V^n \rightarrow V \hspace{1cm} \widetilde{\1}(x_1,
\dots, x_n) = \1$\\
und f\"ur alle $f:V^n \rightarrow V$ Funktionen $f \widetilde{\sqcap}
g:V^n \rightarrow V$, $f \widetilde{\sqcup} g: V^n \rightarrow V$,
$\widetilde{\bar f}: V^n \rightarrow V$ definiert durch:\\
$(f \widetilde{\sqcap} g)(x_1, \dots, x_n) = f(x_1, \dots, x_n) \sqcap
g(x_1, \dots, x_n)$\\
$(f \widetilde{\sqcup} g)(x_1, \dots, x_n) = f(x_1, \dots, x_n) \sqcup
g(x_1, \dots, x_n)$\\
$\widetilde{\bar f}(x_1, \dots, x_n) = \overline{f(x_1, \dots, x_n)}$
\end{ms}

\begin{bw}
Seien $f,g:V^n \rightarrow V$, dann:
\begin{eqnarray*}
&&f \sqcap g = g \sqcap f \\
&\Leftrightarrow& \forall x_1, \dots, x_n \in V: (f \sqcap g)(x_1,
\dots, x_n) = (g \sqcap f)(x_1, \dots, x_n)\\
&\Leftrightarrow& \forall x_1, \dots, x_n) \in V: f (x_1, \dots, x_
n) \sqcap g (x_1, \dots, x_n) = g (x_1, \dots, x_n) \sqcap f(x_1,
\dots, x_n)\\
&\Leftrightarrow& \forall x_1, \dots, x_n \in V:
\underline{\text{wahr}}
\end{eqnarray*}
Rest analog.
\qed
\end{bw}
Die M\"achtigkeit von $V_n$ sei kurz festzuhalten: $\vert V_n \vert =
\vert V\vert^{(\vert V\vert^n)} = (2^a)^b = 2^{a\cdot 2^{a\cdot n}}$\\

Des Weiteren schauen wir uns kurz Boolesche Funktionen auf
$\mathds{B}$ an. Dabei ist $\0: \mathds{B}^n \rightarrow \mathds{B}$
eine Kontraktion und $\1: \mathds{B}^n \rightarrow \mathds{B}$ eine
Tautologie. Weiter gilt $+: \mathds{B} \times \mathds{B} \rightarrow
\mathds{B}$ mit $+ \in \mathds{B}_2$, $\cdot: \mathds{B} \times \mathds{B} \rightarrow
\mathds{B}$ mit $\cdot \in \mathds{B}_2$ und $\bar~: \mathds{B} \rightarrow
\mathds{B}$ mit $\bar~ \in \mathds{B}$. Allgemein gilt $f:
\mathds{B}^n \rightarrow \mathds{B}, f(x_1, \dots, x_n) = t$, wobei
$t$ mittels $x_1, \dots, x_n, +, \cdot$ und $\bar~$ aufgebaut ist.

\begin{mdef}
Seien $n \geq 1$ und $x:=(x_1, \dots, x_n), a:= (a_1, \dots, a_n)$
Vektoren aus $\mathds{B}^n$. Dann definiert man:
\begin{enumerate}
\item $x_i^{a_i} = x_i$, falls $a_i = 1$ (positives Literal)
\item $x_i^{a_i} = \overline{x_i}$, falls $a_i = 0$ (negatives
  Literal)
\item $m_a(x) = \prod_{i=1}^{n} x_i^{a_i}$ ist Minterm $x$ bzgl. $a$
\end{enumerate}
\qed
\end{mdef}

\begin{ms}
F\"ur alle $f \in \mathds{B}_n$ mit $n \geq 1$ und $f \not = \0$
gilt:
\begin{equation*}
f(x_1, \dots, x_n) = \sum_{a \in f^{-1}(1)} m_a(x) \hspace{1cm}
  \text{DNF}
\end{equation*}
Falls $f(x_1, \dots, x_n) = \sum_{a\in f^{-1}(1)} m_a(x) = \0$ definiert, dann ist $f \not = 0$ nicht wahr.
\qed
\end{ms}
\begin{mex}
$f: \mathds{B}^3 \rightarrow \mathds{B}, f(x,y,z)
\mathrel{\widehat{=}}$ "$x+y+z$" gerade

\begin{table}[h]
\begin{tabular}{c|cccccccc}
$x$ & 0 &0 &0 &0 &1 & 1 & 1 &1 \\
$y$ &0 & 0 & 1 & 1 & 0 & 0 &1 & 1\\
$z$ & 0 & 1 & 0 & 1 & 0 & 1 & 0 & 1 \\
\hline $f(x,y,z)$ & 1 & 0 & 0 & 1 & 0 & 1 & 1 & 0\\
\end{tabular}
\end{table}
$f^{-1}(1) = \{ (0,0,0), (0,1,1), (1,0,1), (1,1,0)\}$\\
$f(x,y,z) = \bar x \cdot \bar y \cdot \bar z + \bar x \cdot y \cdot z
+ x \cdot \bar y \cdot z + x \cdot y \cdot \bar z$
\end{mex}

Im Folgenden werden wir das Nicod-Nor und Sheffer-Nand verwenden.
\begin{enumerate}
\item $\triangledown \mathds{B} \times \mathds{B} \rightarrow
  \mathds{B}$ mit $x \triangledown y = \overline{x + y}$
\item $\triangle \mathds{B} \times \mathds{B} \rightarrow
  \mathds{B}$ mit $x \triangle y = \overline{x \cdot y}$
\end{enumerate}

\begin{ms}
\begin{enumerate}
\item Jede Boolesche Funktion $f: \mathds{B}^n \rightarrow \mathds{B}$
  ist durch $\triangledown$ (bzw. durch $\triangle$) darstellbar.
\item $\triangledown$ und $\triangle$ sind die einzigen Funktionen aus
  $\mathds{B}_2$, mit denen man alle $f: \mathds{B}^n \rightarrow
  \mathds{B}$ darstellen kann.
\end{enumerate}
\end{ms}

\begin{bw} ~\\
(1) Es gen\"ugt $+, \cdot, \bar~$ darzustellen, dann folgt die
Behauptung aus Satz 1.3.5 (DNF).
\begin{eqnarray}
x+y &=& \overline{\overline{x + y}} = \overline{\bar x \cdot \bar y} =
\bar x \triangle \bar y = (x \triangle x) \triangle (y \triangle y)\\
x \cdot y &=& \overline{\overline{x \cdot y}} = \overline{\bar x +
  \bar y} = \overline{x \triangle y} = (x \triangle y) \triangle (x
\triangle y)\\
\bar x &=& \overline{x \cdot x} = x \triangle x
\end{eqnarray}
Analog f\"ur $\triangledown$.\\
(2) Sei $\otimes: \mathds{B} \times \mathds{B} \rightarrow \mathds{B}$,
so dass jede Funktion $f: \mathds{B}^n \rightarrow \mathds{B}$ nur
durch $\otimes$ und Variablen darstellbar ist.\\
$0 \otimes 0 = 0$ erlaubt nicht, $\bar ~$ darzustellen, da $\bar 0 =
1$. \\
$1 \otimes 1 = 1$ erlaubt nicht, $\bar ~$ darzustellen, da $\bar 1
= 0$.\\
Also muss $0 \otimes 0 = 1$ sowie $1 \otimes 1 = 0$ gelten.\\

Die restlichen Auswertungen erfordern eine Fallunterscheidung:\\
Fall 1: $0 \otimes 1 = 1$ und $1 \otimes 0 = 0$, dann w\"are $x \otimes
y = \bar x$\\
Fall 2: $0 \otimes 1 = 0$ und $1 \otimes 0 = 1$, dann w\"are $x \otimes
y = \bar y$\\
Somit k\"onnen Fall 1 und Fall 2 nicht auftreten.\\

Fall 3: $0 \otimes 1 = 0$ und $1 \otimes 0 = 0$, dann w\"are $\otimes =
\triangledown$\\
Fall 4: $0 \otimes 1 = 1$ und $1 \otimes 0 = 1$, dann w\"are $\otimes
= \triangle$\\

Insgesamt sind also folgende Wertetafeln m\"oglich: \\
\begin{tabular}{c|cccc}
$x$ & 0 & 0 & 1 & 1 \\
$y$ & 0 & 1 & 0 & 1 \\
\hline $x \otimes y$ & 1 & 0 & 0 & 0
\end{tabular}
\begin{tabular}{c|cccc}
$x$ & 0 & 0 & 1 & 1 \\
$y$ & 0 & 1 & 0 & 1 \\
\hline $x \otimes y$ & 1 & 1 & 1 & 0
\end{tabular}
\qed
\end{bw}

Sei $X = \{x_1, \dots, x_n\}$ Menge von Booleschen Variablen. 
\begin{enumerate}
\item[a)] Zu jedem $\phi \in \ax$ gibt es eine Boolesche Funktion
  $f_{\phi}: \mathds{B}^n \rightarrow \mathds{B}$, so dass f\"ur alle
  Belegungen $V: X \rightarrow \mathds{B}$ gilt:
\begin{equation*}
f_{\phi}(v(x_1), \dots, v(x_n)) = \underline{val}_v(\phi)
\end{equation*}
\item[b)] Zu jeder Booleschen Funktion $f: \mathds{B}^n \rightarrow
  \mathds{B}$ gibt es $\phi_f \in \ax$, so dass f\"ur alle Belegungen
  $V: X \rightarrow \mathds{B}$ gilt:
\begin{equation*}
f(v(x_1), \dots, v(x_n)) = \underline{val}_v(\phi_f)
\end{equation*}
\end{enumerate}

\begin{mex}
Sei $X = \{x, y, z\}$.\\
$\phi \in \ax: \neg x \vee \neg (\neg y \vee z) \vee \neg x$\\
Durch Aufstellen der Wertetafel ergibt sie wie folgt:
\begin{tabular}{c|cccccccc}
$a$ & 0 & 0 & 0 & 0 & 1& 1 & 1 &1 \\
$b$ & 0 & 0 & 1 & 1 & 0 & 0 &1 &1 \\
$c$ & 0 & 1 &0 & 1 & 0 & 1 & 0 & 1\\
\hline $f_{\phi}(a,b,c)$ & 1 & 1 & 1 & 1 & 0 & 0  & 1 & 0 
\end{tabular}

Und daraus ergibt sich dann die Funtkion $f_{\phi}(a,b,c)= \bar a + (\overline{\bar b + c})+ \bar a = \bar a +
b \cdot \bar c$.\\

Sei $V: X \rightarrow \mathds{B}$ Belegung. Dann
\begin{eqnarray*}
\underline{val}_v(\phi) & = & \underline{val}_v(\neg x \vee \neg (\neg
y \vee z) \vee \neg x) \\
& = & \underline{val}_v(\neg x) + \val(\neg (\neg y \vee z)) +
\val(\neg x) \\
& = & \overline{v(x)} + \val(\overline{\neg y \vee z}) +
\overline{v(z)} \\
& = & \overline{v(x)} + \overline{\overline{v(y)} + v(z)} +
\overline{v(x)} \\
& = & f_{\phi}(v(x), v(y), v(z))
\end{eqnarray*}
\end{mex}

\underline{Grundprobleme}
\begin{enumerate}
\item Erf\"ullbarkeit: Gegeben sei $\phi \in \ax$, gibt es $V: X
  \rightarrow \mathds{B}$ mit $\val(\phi) = 1$\\
\"Aquivalent dazu: Gilt $f_{\phi} \not = \0$
\item Berechnung: Gegeben sei $\phi \in \ax$ und \bel. Bestimme
  $\val(\phi)$.\\
\"Aquivalent dazu: $f_{\phi}(v(x_1), \dots, v(x_n))$
\item \"Aquivalenztest: Gegeben sei $\phi, \psi \in \ax$, gilt $\phi
  \Leftrightarrow \psi$\\
\"Aquivalent dazu: $f_{\phi} = f_{\psi}$
\item Erf\"ullende Belegung: Gegeben sei $\phi \in \ax$, bestimme $\{v
  \in \mathds{B}^X \vert \val{\phi} = 1\}$\\
\"Aquivalent dazu: $f_{\phi}^{-1}(1)$
\item Anzahl der erf\"ullenden Belegungen: Gegeben sei $\phi \in \ax$,
  bestimme $\vert \{v \in \mathds{B}^X \vert \val(\phi) = 1\}
  \vert$\\
\"Aquivalent dazu: $\vert f_{\phi}^{-1}(1) \vert$
\end{enumerate}

\subsection{Einige spezielle Konstruktionen}
Definiere wie folgt: $\mathds{B}_n$ Menge, $f: \mathds{B}^n \rightarrow \mathds{B}$\\
$\0: \bol$, $\1: \bol$\\
$\sqcap: \mathds{B}_n \times \mathds{B}_n \rightarrow \mathds{B}_n$,
$\sqcup: \mathds{B}_n \times \mathds{B}_n \rightarrow \mathds{B}_n$,
$\bar ~: \mathds{B}_n \rightarrow \mathds{B}_n$.\\
Dann ist $(\mathds{B}_n, \sqcap, \sqcup, \bar ~, \0, \1)$ eine
Boolesche Algebra.

\begin{mdef}
Es sei \bolf Boolesche Funktion. Eine Varibale $x_i$ hei\ss t
\underline{wesentlich}\footnote{auch: Stelle $i$ in $f$ ist wesentlich.}, falls es $a_1, \dots, a_{n-1} \in \mathds{B}$
gibt mit $f(a_1, \dots, a_{i-1}, 1, a_i, \dots, a_{n-1}) \not = f(a_1,
\dots, a_{i-1}, 0, a_i, \dots, a_{n-1})$.
\qed
\end{mdef}

Man kann mit Hilfe von zwei Schritten, feststellen, welche Variablen
wesentlich sind:
\begin{enumerate}
\item Konstruktion einer Formel $\phi_f \in \ax$ zu $f$
\item \"Aquivalenzumformungen, um Variablen zu entfernen
\end{enumerate}
Wesentliche Variablen werden dabei nicht entfernt, bleiben also
stehen.

\begin{mex}
Sei $X = \{x ,y ,z \} $ und $f: \mathds{B}^3 \rightarrow \mathds{B},
f(x,y,z) = x + y \cdot x + z \cdot x$\\
Formel $\phi_f: x \vee (y \wedge x) \vee (z \wedge x)$
\begin{eqnarray*}
&&x \vee (y \wedge x) \vee (z \wedge x) \\
&\text{logisch \"aquivalent zu:}& x \vee (z \wedge x) \hspace{1cm} \text{Absorption}\\
&\text{logisch \"aquivalent zu:}& x \hspace{2,55cm} \text{Absorption}
\end{eqnarray*}
Also $f(a,b,c) = a$  f\"ur alle $a,b,c \in \mathds{B}$. Das hei\ss t
$x$ ist wesentlich, $y$ und $z$ hingegen sind unwesentlich.
\end{mex}

\begin{mdef}
Es sei \bolf eine Boolesche Funktion. Zu $i, 1 \leq i \leq n$ ist der
\underline{positive Co-Faktor} definiert als
\begin{equation*}
f_{x_i}: \mathds{B}^n \rightarrow \mathds{B}^n, f_{x_i}(y_1, \dots,
y_n)= f(y_1, \dots, y_{i-1}, 1, y_{i+1}, \dots, y_n)
\end{equation*}
und der \underline{negative Co-Faktor} definiert als
\begin{equation*}
f_{\overline{x_i}}: \mathds{B}^n \rightarrow \mathds{B}^n,
f_{\overline{x_i}}(y_1, \dots, y_n) = f(y_1, y_{i-1},
0, y_{i+1}, \dots, y_n).
\end{equation*}
\qed
\end{mdef}

Im Folgenden wird die \underline{Shannon-Zerlegung} verwendet:
\begin{equation*} f = x_i \cdot f_{x_i} + \overline{x_i} \cdot
  f_{\overline{x_i}} \end{equation*}

\begin{mdef}
Die Projektionsfunktion $x_i: \bol$ ist f\"ur $i, 1 \leq i \leq n$
definiert durch $x_i(y_1, \dots, y_n) = y_i$.\\
\qed
\end{mdef}

F\"ur die Operatoren der Booleschen Algebra bedeutet das wie folgt:\\
\begin{enumerate}
\item $x_i \cdot f_{x_i} \mathrel{\widehat{=}} x_i \sqcap f_{x_i}$ mit
$\sqcap: \mathds{B}_n \times \mathds{B}_n \rightarrow \mathds{B}_n$
\item $g + h \mathrel{\widehat{=}} g \sqcup h$ mit
$\sqcup: \mathds{B}_n \times \mathds{B}_n \rightarrow \mathds{B}_n$
\item $\bar ~ \mathrel{\widehat{=}} \bar ~: \mathds{B}_n
  \rightarrow \mathds{B}_n$
\end{enumerate}

\begin{ms}
F\"ur alle $f \in \mathds{B}_n$ und alle $i, 1 \leq i \leq n$ gilt
\begin{equation*}
f = x_i \cdot f_{x_i} + \overline{x_i} \cdot f_{\overline{x_i}}.
\end{equation*}
\end{ms}

\begin{bw}
Sei $(y_1, \dots, y_n) \in \mathds{B}^n$ beliebig.\\
Zu zeigen: $f(y_1, \dots, y_n) = (x_i \cdot f_{x_i} + \overline{x_i}
\cdot f_{\overline{x_i}})(y_1, \dots, y_n)$.
\begin{eqnarray*}
&&(x_i \cdot f_{x_i} + \overline{x_i} \cdot f_{\overline{x_i}})(y_1,
\dots, y_n)\\
&=& (x_i \cdot f_{x_i})(y_1, \dots,
y_n)+(\overline{x_i} \cdot f_{\overline{x_i}})(y_1, \dots, y_n)\\
&=& x_i(y_1, \dots, y_n) \cdot f_{x_i}(y_1, \dots, y_n) +
\overline{x_i}(y_1, \dots, y_n) \cdot f_{\overline{x_i}}(y_1, \dots,
y_n)\\
&=& y_i \cdot f(y_1, \dots, y_{i-1}, 1, y_{i+1}, \dots, y_n) +
\overline{x_i}(y_1, \dots, y_n) \cdot f_{\overline{x_i}}(y_1, \dots,
y_n)\\
&=& y_i \cdot f(y_1, \dots, y_{i-1}, 1, y_{i+1}, \dots,y_n) +
\overline{y_i} \cdot f(y_1, y_{i-1}, 0, y_{i+1}, \dots, y_n) 
\end{eqnarray*}

Fall 1: $y_i = 1$

\begin{equation*}
(x_i \cdot f_{x_i} + \overline{x_i} \cdot
f_{\overline{x_i}})(y_1,\dots, y_n) = f(y_1,\dots, y_i, \dots, y_n) +
0 = f(y_1, \dots, y_n)
\end{equation*}

Fall 2: $y_i = 0$
\begin{equation*}
(x_i \cdot f_{x_i} + \overline{x_i} \cdot
f_{\overline{x_i}})(y_1,\dots, y_n) = 0+ f(y_1,\dots, y_i, \dots, y_n) = f(y_1, \dots, y_n)
\end{equation*}
\qed
\end{bw}

\begin{mdef}
Es sei \bolf, zu $i$, $1 \leq i \leq n$ hei\ss t die Funktion 
\begin{equation*}
\exists x_i f: \mathds{B}^n \rightarrow \mathds{B}, \hspace{0.5cm}
\exists x_i f= f_{x_i} + f_{\overline{x_i}}
\end{equation*}
die \underline{Existenzialquantifizierung} und 
\begin{equation*}
\forall x_i f: \mathds{B}^n \rightarrow \mathds{B}, \hspace{0.5cm}
\forall x_i f= f_{x_i} \cdot f_{\overline{x_i}}
\end{equation*}
die \underline{Allquantifizierung} von $f$ nach $x_i$.
\qed
\end{mdef}

\begin{ms}
Es seien $f: \mathds{B}^n \rightarrow \mathds{B}^n$ und $i$, $1 \leq i
\leq n$. Dann gilt f\"ur alle $(y_1,\dots,y_n) \in \mathds{B}$
\begin{equation*}
(\exists x_i f)(y_1,\dots,y_n) =1 \Leftrightarrow \exists a \in
\mathds{B}: f(y_1,\dots,y_{i-1},a,y_{i+1},\dots, y_n) = 1
\end{equation*}
\begin{equation*}
(\forall x_i f)(y_1,\dots,y_n) =1 \Leftrightarrow \forall a \in
\mathds{B}: f(y_1,\dots,y_{i-1},a,y_{i+1},\dots, y_n) = 1
\end{equation*}
\end{ms}

\begin{bw}
F\"ur $\exists x_i f$:
\begin{eqnarray*}
(\exists x_i f)(y_1, \dots, y_n) &=& (f_{x_i} +
f_{\overline{x_i}})(y_1, \dots, y_n)\\
&=& f_{x_i}(y_1,\dots,y_n) + f_{\overline{x_i}}(y_1,\dots,y_n) \\
&=& y_i \cdot f_{x_i}(y_1,\dots,y_{i-1},1,y_{i+1},\dots,y_n) +
\overline{y_i} \cdot f_{\overline{x_i}}(y_1,\dots,
y_{i-1},0,y_{i+1},\dots, y_n)\\
(\exists x_i f)(y_1,\dots, y_n) = 1 &\Leftrightarrow& y_i \cdot
f(y_1,\dots, 1, \dots, y_n) + \overline{y_i} \cdot
f(y_1,\dots,0,\dots,y_n) =1 \\
&& y_i = 1 \text{ oder } y_i = 0 \\
&\Leftrightarrow& f(y_1,\dots,1,\dots,y_n) =1 \text{ oder }
f(y_1,\dots, 0, \dots, y_n) = 1
\end{eqnarray*}
\qed
\end{bw}

%%%%%%%%%%%%%%%%%%%%%%%
%%%%%%%%%%%%%%%%%%%%%%%
\newpage
%%%%%%%%%%%%%%%%%%%%%%%
%%%%%%%%%%%%%%%%%%%%%%%

\section{Bin\"are Entscheidungsdiagramme}
Folgende Themen sollen behandelt werden:
\begin{itemize}
\item Grundlagen Graphentheorie 
\item Grundlagen BDDs
\item BDDs und Boolesche Funktionen
\item BDDs f\"ur spezielle Funktionen
\end{itemize}

\subsection{Graphentheoretische Grundlagen}
Es soll die \"ubliche Konvention gelten:
$g=(V,P), V \not = \emptyset, \vert V \vert < \infty \text{ Knotenmenge },
P \subseteq V \times V \text{ Pfeilmenge }, (x,y) \in P, x,y \in V$.

\begin{mdef}
Ein gerichteter (Multi-) Graph ist ein 4-Tupel $g=(V,P,\alpha,
\omega)$ mit 
\begin{enumerate}
\item[(1)] $V$ endlich und $V \not = \emptyset, x \in V$ hei\ss t
  \underline{Knoten}
\item[(2)] $P$ endliche Menge, $p \in P$ hei\ss t \underline{Pfeil}
\item[(3)] $\alpha, \omega: P \rightarrow V$ sind Funktionen, dabei
  hei\ss en $\alpha(p)$ und $\omega(p)$ \underline{Anfangs- bzw. Endknoten} von $p
  \in P$.
\end{enumerate}
Gilt f\"ur $x,y \in V$ und $p \in P$, dass $\alpha(p) = x$ und
$\omega(p)=y$, so hei\ss t $x$ \underline{Vorg\"anger} von $y$ und $y$ \underline{Nachfolger}
von $x$.\\
\begin{enumerate}
\item[(4)] $d^{+}g(x) = \vert \{ p \in P ~\vert~ \alpha(p)=x\}\vert$
  hei\ss t Au\ss engrad von $x \in V$
\item[(5)] $d^{-}g(x) = \vert \{ p \in P ~\vert~ \omega(p)=x\}\vert$
  hei\ss t Innengrad von $x \in V$
\end{enumerate}
\qed
\end{mdef}

\begin{mdef}
Sei $g=(V,P,\alpha, \omega)$ ein Graph.
\begin{enumerate}
\item[(1)] Eine Folge $(p_1,\dots,p_n) \in P^+$\footnote{nichtleere
    Liste bzw $n \geq 1$} hei\ss t \underline{Weg} von $\alpha{(p_1)}$
  nach $\omega(p_n)$, falls f\"ur alle $i, 1 \leq i \leq n$ gilt:
  $\omega(p_i) = \alpha(p_{i+1})$. Sind alle paarweise verschieden, so
  hei\ss t der Weg \underline{einfach}.
\item[(2)] Ein Weg $(p_1, \dots, p_n)$ hei\ss t \underline{Kreis}, falls
  $\omega(p_n) = \alpha(p_1)$. Ein Kreis hei\ss t \underline{einfach},
  falls alle Pfeile paarweise verschieden sind.
\item[(3)] Knotenlisten von Wegen nach Pfeilen\footnote{Pfeile sind
    eindeutig} $(p_1, \dots, p_n)$ lassen sich wie folgt darstellen:
  $(\alpha(p_1),\dots, \alpha(p_n), \omega(p_n)$.
\item[(4)] Seien $x, y \in V$. Dann hei\ss t $y$ "\underline{von $x$
    aus erreichbar}", falls $x=y$ oder ein Weg $(p_1, \dots, p_n) \in
  P^+$ mit $\alpha(p_1) = x$ und $\omega(p_n) = y$ existiert.
\item[(5)] $g$ hei\ss t \underline{kreisfrei}, falls es keinen Kreis
  gibt.
\end{enumerate}
\qed 
\end{mdef}

\begin{mex}
BILD\\
Die Wegl\"ange entspricht der Anzahl der Pfeile.
\end{mex}

\begin{mdef}
Seien $g = (V,P,\alpha, \omega)$ Graph und $x \in V$. Dann hei\ss t
$x$
\begin{enumerate}
\item[(1)] \underline{Quelle}, falls $d^-g(x)=0$ ($\omega(p) \not = x$
  f\"ur alle $p \in P$)
\item[(2)] \underline{Senke}, falls $d^+g(x)=0$ ($\alpha(p) \not = x$
  f\"ur alle $p \in P$)
\item[(3)] \underline{Wurzel}, falls jeder Knoten $y \in V$ von $x$
  aus erreichbar ist.
\end{enumerate}
\qed
\end{mdef}

\begin{mdef}
Sei $g = (V,P, \alpha, \omega)$ Graph. Dann hei\ss t g
\begin{enumerate}
\item[(1)] \underline{Knotenmarkiert}, falls es eine Funktion $m: V
  \rightarrow M$ gibt ($m(x)$ hei\ss t \underline{Marke} von $x \in
  V$).
\item[(2)] \underline{Pfeilmarkiert}, falls es eine Funktion $m: P
  \rightarrow M$ gibt ($m(x)$ hei\ss t \underline{Marke} von $p \in P$).
\end{enumerate}
\qed
\end{mdef}

\subsection{Grundlagen BDDs}

\begin{mex} \label{mex221}
Hier k\"onnte Ihre Werbung stehen!
\end{mex}


\begin{mdef}
Es sei $X = \{x_1, \dots. x_n\}$ eine Menge von Variablen. Ein
\underline{BDD} zu $X$ ist ein gerichteter Graph $g =
(V,P,\alpha,\omega)$ mit den folgenden Eigenschaften
\begin{enumerate}
\item[(1)] Der Graph g ist kreistfrei, hat genau eine Wurzel und genau
  zwei Senken, genannt $\mathds{I}$ und $\mathds{O}$.
\item[(2)] Jeder Knoten ungleich $\mathds{I}$ und $\mathds{O}$ hei\ss
  t \underline{innerer Knoten} und tr\"agt eine Variable aus $X$ als
  Marke. $\mathds{I}$ hei\ss t 1-Senke und tr\"agt die 1 als Marke und
  $\mathds{O}$ hei\ss t 0-Senke und tr\"agt die 0 als Marke. Falls $V'
  = V \cup \{\mathds{I}, \mathds{O}\}$, dann $V$ innere Marken.
\item[(3)] Jeder Knoten ungleich 
$\mathds{I}$ und $\mathds{O}$ hat
  genau zwei ausgehende Pfeile. Ein Pfeil ist mit 1 markiert und
  hei\ss t 1-Pfeil, der andere ist mit 0 markiert und hei\ss t
  0-Pfeil. \underline{1- und 0-Nachfolger} von $x \in V$ sind entsprechend definiert.
\item[(4)] F\"ur alle Wege von der Wurzel zu einem Knoten sind die
  Knotenmarkierungen der Knoten der Knotenlisten paarweise
  verschieden.
\end{enumerate}
\qed
\end{mdef}

\begin{mbez}
\begin{enumerate}
\item[(1)] $\mathds{I}$ 1-Senke, $\mathds{O}$ 0-Senke.
\item[(2)] F\"ur $x \in V$ ist $\underline{var}(x) \in X$
  \underline{Markierung} ($\underline{var}(\mathds{I})=1$ und
  $\underline{var}(\mathds{O})=0$).
\item[(3)] $\underline{size}(g)=\vert V \vert$ ist die Anzahl der
  \underline{inneren Knoten}, $\underline{height}(g)$ bezeichnet die
  L\"ange eines l\"angsten Weges von der Wurzel zu einer
  Senke\footnote{H\"ohe von $g$}, $\underline{width}(g)= \max_{i \in
    \{1,\dots,n\}} \vert V_i \vert$ mit $V_i = \{ a \in V \vert
  \underline{var}(a)=x_i\}$.
\item[(4)]  Zu $a \in V$ hei\ss t ist 1-Nachfolger,
  $\underline{then}(a)$ und 0-Nachfolger $\underline{else}(a)$.
\end{enumerate}
\qed
\end{mbez}

\begin{mdef}
Sei $(X,>)$ Variablenordnung, wobei $>$ lineare Striktordnung auf
$X$. Ein BDD $g$ hei\ss t geordnet oder OBDD, falls f\"ur alle Wege $(p_1,\dots,p_n)$ 
von der Wurzel bis zur einer Senke gilt:
\begin{equation*}
\underline{var}(p_i) < \underline{var}(p_{i+1}) \text{ f\"ur } i, 1
\leq i \leq n-1
\end{equation*}
\qed
\end{mdef}

\begin{ex}
Regelbsp. Entfernen eines redundanten Knotens
\end{ex}

\begin{mr}\label{regel:entf}
Es sei $g=(V,P,\alpha,\omega)$ ein OBDD. Ein innerer Knoten $a \in V$
hei\ss t redundant, falls $\underline{then}(a) = \underline{else}(a)$.
Das Entfernen eines redundanten Knotens $a$ funktioniert wie folgt:
\begin{itemize}
\item[(1)] $a$ wird aus V entfernt.
\item[(2)] Jeder Pfeil $p \in P$ mit $\alpha(p) = a$ oder $\omega(p) =
  a$ wird aus $P$ entfernt.
\item[(3)] F\"ur jeden Vorg\"angerknoten von $a$, f\"uge einen
  \begin{itemize}
  \item[(i)] 1-Pfeil nach $\underline{then}(a)$, falls es ein 1-Pfeil
    war.
  \item[(ii)] 0-Pfeil nach $\underline{else}(a)$, sonst.
  \end{itemize}
\end{itemize}
\qed
\end{mr}

\begin{mdef}
Sei $g=(V,P,\alpha,\omega)$ ein OBDD und $a \in V$ innerer Knoten. Das
durch $a$ induzierte Unter-OBDD $g'=(V',P',\alpha\,\omega')$ ist wie
folgt definiert:
\begin{itemize}
\item[(1)] $V' = \{b \in V ~\vert~ b \text{ von } a \text{ aus erreichbar}\}$
\item[(2)] $P' = \{p \in P ~\vert~ \alpha(p) \in V' \wedge \omega(p) \in
  V'\} $
\item[(3)] $\alpha' : P' \rightarrow V'$ ist definiert durch $\alpha'(p) =
  \alpha(p)$ f\"ur alle $p \in P$.\\
  $\omega' : P' \rightarrow V'$ ist definiert durch $\omega'(p) =
  \omega(p)$ f\"ur alle $p \in P$.
\item[(4)] Jedes $a \in V'$ hat die gleiche Marke wie in g.
\item[(5)] Jeder $p \in P$ hat den selben Typ (0- oder 1-Pfeil) wie in g.
\end{itemize}
\qed
\end{mdef}

\begin{mdef}
Seien $g=(V,P,\alpha,\omega)$ und $g'=(V',P',\alpha',\omega')$
OBDDs. Dann hei\ss en $g$ ung $g'$
\underline{strukturgleich}\footnote{auch: isomorph}, falls es bijektive
Funktionen $\Phi: V \rightarrow V'$ und $\Psi: P \rightarrow P'$ gibt, so dass f\"ur alle $p \in P$ gilt
\begin{itemize}
\item[(1)] $\alpha'(\Psi(p)) = \Phi(\alpha(p))$
\item[(2)] $\omega'(\Psi(p)) = \Phi(\omega(p))$
\item[(3)] $p$ ist 1-Pfeil $\rightarrow \Psi(p)$ ist 1-Pfeil
\item[(4)] $p$ ist 0-Pfeil $\rightarrow \Psi(p)$ ist 0-Pfeil
\end{itemize}
und f\"ur alle $a \in V$ gilt
\begin{itemize}
\item[(5)] $\underline{var}(\Phi(a)) = \underline{var}(a)$
\end{itemize}
\qed
\end{mdef}

\begin{mex}
Hier sollte ein Bild sein.
\end{mex}

\begin{mr}\label{regel:verschm}
Sei $g=(V,P,\alpha,\omega)$ OBDD. Innere Knoten $a,b \in V$ hei\ss en
\underline{\"aquivalent}, falls gilt:
\begin{eqnarray*}
\underline{then}(a) & = & \underline{then}(b)\\
\underline{else}(a) & = & \underline{else}(b)\\
\underline{var}(a) & = & \underline{var}(b)
\end{eqnarray*}
Das Verschmelzen von $a$ und $b$\footnote{durch Entfernen von $b$}
funktioniert wie folgt:
\begin{itemize}
\item[(1)] $b$ wird aus $V$ entfernt.
\item[(2)] Jeder Pfeil $p \in P$ mit $\alpha(p) = b$ bzw. $\omega(p)=b$
  wird aus $P$ entfernt.
\item[(3)] F\"ur alle Vorg\"anger $c$ von $b$ f\"uge
  \begin{itemize}
  \item[(i)] 1-Pfeil von $c$ nach $a$ ein gdw. 1-Pfeil von $c$ nach
    $b$ existierte,
  \item[(ii)] 0-Pfeil von $c$ nach $a$ ein, sonst.
  \end{itemize}
\end{itemize}
\qed
\end{mr}

\begin{mdef}
Sei $g$ ein OBDD \"uber Variablenmenge $X$. Dann hei\ss t $g$
\begin{itemize}
\item[(1)] \underline{vollst\"andig}, falls f\"ur jeden Weg
  $(p_1,\dots,p_n)$ von der Wurzel zu einer Senke ($\mathds{I}$ oder
  $\mathds{O}$) gilt: $n = \vert X \vert$.
\item[(2)] \underline{quasi-reduziert} (kurz: QOBDD), falls es
  vollst\"andig ist und es keine \"aquivalenten Knoten gibt.
\item[(3)] \underline{reduziert} (kurz: ROBDD), falls es keine
  redundanten und \"aquivalenten Knoten besitzt.
\end{itemize}
Ist $\pi = (x_1, \dots, x_n)$ die Variablenordnung, so nennt man
$V_i=\{a \in V ~\vert~ \underline{var}(a)=x_i\}$ die $i$-te
\underline{Schicht}, wobei $1 \leq i \leq n$ und $\{\mathds{I},
\mathds{O}\}$ hei\ss t die $n+1$-te Schicht.\\
\qed
\end{mdef}

\begin{mex}
Ein weiteres Beispiel
\end{mex}

\subsection{Darstellung von Booleschen Funktionen durch BDDs}
\begin{mdef}
Sei $g=(V,P,\alpha,\omega)$ OBDD.
\begin{equation*}
f_{\mathds{I}}: \mathds{B}^n \rightarrow \mathds{B}, \hspace{1cm}
f_{\mathds{I}}(x_1,\dots, x_n)=1
\end{equation*}
\begin{equation*}
f_{\mathds{O}}: \mathds{B}^n \rightarrow \mathds{B}, \hspace{1cm}
f_{\mathds{O}}(x_1,\dots, x_n)=0
\end{equation*}
und f\"ur jeden inneren Knoten $a \in V$ ist
\begin{equation*}
f_a : \mathds{B}^n \rightarrow \mathds{B}, \hspace{1cm}
f_a(x_1,\dots,x_n) = x \cdot f_{then(a)}(x_1,\dots,x_n) + \overline{x}
\cdot f_{else(a)}(x_1,\dots,x_n)
\end{equation*}
mit $x:=\underline{var}(a)$ definiert. Ist $w$ Wurzel von $g$, so ist
$f_w$ die von $g$ dargestellte Funktion.
\qed
\end{mdef}

\begin{mex}
Sei $X=\{x,y,z,u\}$ und $\pi=(x,y,z,u)$. \\
Hier fehlt ein Bild.\\
\begin{eqnarray*}
f_a(x,y,z,u) &=& x \cdot f_c(x,y,z,u) + \overline{x} \cdot
f_b(x,y,z,u) \\
&= & x \cdot (z \cdot f_d(x,y,z,u) + \overline{z} \cdot f_{\mathds{O}}(x,y,z,u))
+ \overline{x} \cdot (y \cdot f_c(x,y,z,u) + \overline{y} \cdot
f_{\mathds{O}}(x,y,z,u))\\
&=& x \cdot (z \cdot (u \cdot f_{\mathds{I}}(x,y,z,u) + \overline{u} \cdot
f_{\mathds{O}}(x,y,z,u)) + \overline{z} \cdot 0) + \\ && \overline{x} \cdot
(y \cdot (z \cdot (u \cdot f_{\mathds{I}}(x,y,z,u) + \overline{u}
\cdot f_{\mathds{O}}(x,y,z,u)) + \overline{z} \cdot 0)) \\
&=& x \cdot (z \cdot (u \cdot 1 + \overline{u} \cdot 0)) +
\overline{x} \cdot (y \cdot (z \cdot (u \cdot 1 + \overline{u} \cdot
0)))\\
&=& x \cdot z \cdot u + \overline{x} \cdot y \cdot z \cdot u 
\end{eqnarray*}
\end{mex}

\begin{ms}
Sei $g$ ein OBDD. Zu einem Weg $(p_1,\dots,p_n)$ von der Wurzel $w$
bis zur $\mathds{I}$ (1-Senke) sei definiert:
\begin{equation*}
m(p_1,\dots,p_k) = \prod_{i=1}^{k}
\underline{var}(\alpha(p_i))^{a_i}
\end{equation*}
wobei $x_i^{a_i}$ wie in Def. 1.3.4 und $a_i=1$ gdw. $p_i$ 1-Pfeil bzw. $a_i=0$ gdw. $p_i$
0-Pfeil ist, dann gilt 
\begin{equation*} f_w(x_1,\dots,x_n) = \sum_{w \in \mathcal{W}}
m(w)
\end{equation*}
wobei $\mathcal{W}$ die Menge der Wege von $w$ nach
$\mathds{I}$ ist.
\end{ms}

\begin{bw}
Induktion \"uber die H\"ohe des OBDDs\\
\underline{IA:} H\"ohe sei 1.\\
Hier fehlen wieder Bilder.\\
1. Fall: \\
\begin{eqnarray*}
f_w(x_1,\dots,x_n)= x \cdot f_{\mathds{I}}(x_1,\dots,x_n) +
\overline{x} \cdot f_{\mathds{O}}(x_1,\dots,x_n) = x
\end{eqnarray*}
Dann gilt $\mathcal{W}=\{(p)\}$, $m((p)) = x$ und schlie\ss lich
$\sum_{w \in \mathcal{W}} m(w) = m((p))=x = f_w(x_1,\dots,x_n)$.

Analog f\"ur Fall 2 bis 4.
\underline{IS:} Sei H\"ohe $> 1$ und $g$ habe folgende Form (Bild).\\
\begin{eqnarray*}
f_a(x_1,\dots,x_n) &=& x \cdot f_b(x_1,\dots,x_n) + \overline{x} \cdot
f_c(x_1,\dots,x_n) \\
&=& x \cdot \sum_{w \in \mathcal{W}_1} m(w) + \overline{x} \cdot
\sum_{v \in \mathcal{W}_2} m(v) \\
&=& \sum_{w \in \mathcal{W}_1} x \cdot m(w) + \sum_{v \in
  \mathcal{W}_2} \overline{x} \cdot m(v) \\
&=& \sum_{w \in \mathcal{W}, w_1 = p} m(w) + \sum_{v \in
  \mathcal{W}, v_1 = q} m(v)\\
&=& \sum_{w \in \mathcal{W}} m(w)
\end{eqnarray*}
\qed
\end{bw}

\begin{mk}
Bei ROBDDs werden auch die 1- und 0-Senke als OBDDs aufgefasst, mit
$\mathds{I}$ bzw. $\mathds{O}$ bezeichnet und es wird $f_{\mathds{I}}$
und $f_{\mathds{O}}$ wie in Def. 2.3.1 definiert.
\qed
\end{mk}

\begin{ms}
Zu jeder Booleschen Funktion existiert ein OBDD, welcher die Funktion
darstellt. Dies gilt auch, wenn man QOBDDs bzw. ROBDDs betrachtet.
\qed
\end{ms}

\begin{ms}
Es sei $f$ eine Boolesche Funktion. Wird $f$ durch zwei ROBDDs (oder
QOBDDs) $g_1$ und $g_2$ dargestellt, so sind $g_1$ und $g_2$ strukturgleich.
\qed
\end{ms}

Es ist festzuhalten, dass die Variablenordnung Einfluss auf die
Gr\"o\ss e hat.

\begin{mex}
BILD
\end{mex}

\subsection{OBDDs f\"ur wichtige Funktionen}
Gaaaanz viele Bilder.

%%%%%%%%%%%%%%%%%%
%%%%%%%%%%%%%%%%%%
\newpage
%%%%%%%%%%%%%%%%%%
%%%%%%%%%%%%%%%%%%

\section{Algorithmen}
\begin{itemize}
\item Minimierung
\item Konstruktion (Zusammenbau, Synthese)
\item Operation f\"ur
Basisfragen von logischen Formeln
\end{itemize}

\subsection{Minimierung}
Gegeben: OBDD\\
Aufgabe: Konstruiere einen ROBDD durch Anwenden der Regeln
\hyperref[regel:entf]{\glqq Entfernen redundanter Knoten\grqq} und
\hyperref[regel:verschm]{\glqq Verschmelzen \"aquivalenter Knoten\grqq} soweit wie
  m\"oglich.

\begin{obdd}
Sei $g$ ein OBDD. Jeder Knoten $a$ wird als \emph{Record} mit $4$
Komponenten dargestellt.
\begin{itemize}
\item[(1)] a.id - Knotenbezeichnung, nat\"urliche Zahl (alle
  verschieden)
\item[(2)] a.var - Beschriftung, d.h. Variable\footnote{Index der
    Variable bei Implementierung}
\item[(3)] a.then - Verweis auf $then(a)$, falls $a$ innerer Knoten,
  sonst $nil$
\item[(4)] a.else - Verweis auf $else(a)$, falls $a$ innerer Knoten,
  sonst $nil$
\end{itemize}
Zus\"atzlich: Mengen $S_1,\dots,S_n$ mit $S_i=\{ a \vert
\underline{var}(a)=x_i\}$. Im konkreten $S_i$ gibt es eine Liste von
Verweise auf Records mit $a.var = x_i$ (bzw. $a.var = i$).
\end{obdd}

\begin{mex}
Bildchen
\end{mex}

\begin{algred}
  Eingabe: OBDD $g$ zu einer Variablenordnung (in
  Zeigergeflechtdarstellung)\\
  Ausgabe: ROBDD - Geflechtdarstellung zu $g$ \\

  \texttt{Reduce}(g):
  \begin{itemize}
  \item[(1)] Durchlaufe g in DFS-Ordnung und setze dabei\\
      \lstinline{0-Senke.id := 0}\\
      \lstinline{1-Senke.id := 1}\\
      \lstinline{z[0] = 0-Senke}\\
      \lstinline{z[1] = 1-Senke}
 \item[(2)] \lstinline{k := 2}\\
      \lstinline{for i=n-1 until 1 do}
      \begin{itemize}  
      \item[(1)] Suche in $S_i$ alle redundanten Knoten und markiere sie als
entfernbar.
      \item[(2)] Sortiere die Mengen $S_i$ nach der folgenden Ordnung:\\
      $a \sqsubseteq b \Leftrightarrow$ (\lstinline{a.then.id, a.else.id})
           $\leq_{lex.}$ (\lstinline{b.then.id, b.else.id})\\
      Damit stehen \"aquivalente Knoten in der Liste immer
      hintereinander.
      \item[(3)] Suche die verschmelzbaren Knoten und markiere alle
        bis auf einen als entfernbar. F\"ur jede Teilliste
        $a_1,\dots,a_n$ von \"aquivalenten Knoten wird $a_1$
        Vertreterknoten und $a_2,\dots,a_n$ werden als entfernbar
        markiert.\\
        \lstinline{k := k+1}\\
        \lstinline{a1.id = k}\\
        \lstinline{z[k] = }$a_1$
        \item[(4)] Alle in (1) und (3) als entferbar markierten Knoten
          werden entfernt und auf sie verweisende Zeiger mithilfe von
          $z$ entsprechend umgelenkt.
    \end{itemize}
  \end{itemize}  
\end{algred}

\subsection{Auswertung und Erf\"ullbarkeit}
Auswertung:\\
Gegeben sei ein OBDD zur Funktion \bolf ~und $b_1,\dots\,b_n \in
\mathds{B}^n$. Bestimme $f(b_1,\dots,b_n)$.\\

Erf\"ullbarkeit:\\
Gegeben sei ein OBDD zur Funktion \bolf. Gibt es $(b_1,\dots,b_n) \in
\mathds{B}^n$ mit $f(b_1,\dots,b_n)=1$?

\begin{algval}
  Eingabe: OBDD $g$ zu $f$ und $(b_1,\dots,b_n) \in \mathds{B}^n$.\\

  \texttt{Value(g,b)}
  \begin{minipage}{0.5\textwidth}
  \begin{empheq}[box=\fbox]{align*}
    &a := g\\
    &\text{while a keine Senke do }\\
    &\text{\hspace{0.9cm} if } b[a.var] = 0 \text{ then } a := a.else\\
    &\text{\hspace{0.9cm} else } a := a.then\\
    &\text{return } a.var
  \end{empheq}
\end{minipage}
\end{algval}

\begin{mex}
  $X = \{x_1,x_2,x_3,x_4\}, ~\pi = (x_1,x_2,x_3,x_4), ~ b= (0,1,0,1)$\\
  $f(x)= x_1 \cdot x_2 \cdot x_3 \cdot x_4 + x_1 \cdot x_2 \cdot
  \overline{x_3} \cdot x_4 + x_1 \cdot \overline{x_2} \cdot x_3 \cdot
  x_4 + \overline{x_1} \cdot x_2 \cdot \overline{x_3} \cdot x_4 +
  \overline{x_1} \cdot x_2 \cdot x_3 \cdot x_4 + \overline{x_1} \cdot
  \overline{x_2} \cdot x_3 \cdot x_4$
\end{mex}

\begin{algsat}
Eingabe: OBDD $g$ und globales Feld $B$\\
Ausgabe: $1$ und Belegung von $B$, die den Wert $1$ liefert, sonst
$0$\\

\texttt{Satisfy(g,B)}\footnote{Komplexit\"at:
  $O(\underline{size}(g))$}
\begin{minipage}{0.5\textwidth}
\begin{empheq}[box=\fbox]{align*}
 &a := g\\
 &\text{if } a \text{ Senke then return } a.var\\
 &\text{else } B[a.var] = 1\\
 &\text{if } Satisfy(a.then, B) \text{ then return } 1\\
 &B[a.var] := 0\\
 &\text{if } Satisfy(a.else, B) \text{ then return } 1
\end{empheq}
\end{minipage}
\end{algsat}

\begin{mex}
  Sei $X = \{x_1,x_2,x_3\}$ und $\pi = (x_1,x_2,x_3)$, sowie $f(x) =
  x_1 \cdot \overline{x_2} \cdot x_3 = \overline{x_1} \cdot
  \overline{x_2} \cdot x_3$
\end{mex}

\begin{ms}\label{sat:anz}
  Sei \bolf ~Boolesche Funktion, die durch ein QOBDD dargestellt
  wird. Dann ist die Anzahl der erf\"ullenden Belegungen gegeben durch
  $\underline{anz}(a)$, wobei anz rekursiv wie folgt definiert
  ist\footnote{F\"ur ROBDDs gilt dies jedoch nicht.}:
  \begin{align*}
    \underline{anz}(\mathds{O}) = & ~0\\
    \underline{anz}(\mathds{I}) = & ~1\\
    \underline{anz}(a) = & ~\underline{anz}(a.then) +
    \underline{anz}(a.else), \text{ wobei } a \text{ keine Senke }
  \end{align*}
\qed
\end{ms}

\begin{mex}
QOBDD zur Funktion von \hyperref[mex221]{Beispiel \ref{mex221}}\\
$X = \{x_1,x_2,x_3\}, ~\pi = (x_1,x_2,x_3)$
\end{mex}

Sei $g$ ein ROBDD \"uber $X = \{x_1, \dots, x_n\}$. F\"ur Knoten $a$
von $g$ wird $\underline{level}(a)$ definiert als 
\begin{itemize}
\item Index von $\underline{var}(a)$, falls $a$ innerer Knoten
\item $n+1$, falls $a$ Senke
\end{itemize}

\begin{ms}
  Sei $g$ ein ROBDD, welches Funktion \bolf ~darstellt. Dann ist die
  Anzhal der erf\"ullenden Belegungen von $f$ gegeben durch
  $\underline{anz}(g)$\footnote{wenn g Wurzel ist}, wobei die Funktion
  anz auf ROBDD Knoten definiert ist durch:
  \begin{align*}
    \underline{anz}(\mathds{I})=& ~1\\
    \underline{anz}(\mathds{O})=& ~0\\
    \underline{anz}(a) =& ~\frac{1}{2} \cdot \biggr(
  \underline{anz}(\underline{then}(a)) \cdot
  2^{~\underline{level}(\underline{then}(a)) - \underline{level}(a)} +
  \underline{anz}(\underline{else}(a)) \cdot
  2^{~\underline{level}(\underline{else}(a)) - \underline{level}(a)} \biggr)
  \end{align*}
\qed
\end{ms}

\begin{mex}
Boolesche Addition.\\

ROBDD zur Funktion von \hyperref[mex221]{Beispiel \ref{mex221}}.
\end{mex}

\subsection{\"Aquivalenztest}
Problem: $f_1, f_2 : \mathds{B}^n \rightarrow \mathds{B}$. Gilt $f_1 =
f_2$? (D.h. $f_1(a_1,\dots,a_n) = f_2(a_1,\dots,a_n)$
f.a. $(a_1,\dots,a_n) \in \mathds{B}^n$)\\
L\"osung: Formuliere $f_1$ und $f_2$ \"uber gleichen Variablen $X =
\{x_1,\dots, x_n\}$. Lege (m\"oglichst g\"unstige) Variablenordnung
fest und stelle $f_1$ und $f_2$ durch BDDs\footnote{am besten sogar
  als ROBDDs} $g_1, g_2$ dar.\\
Dann gilt: $f_1 = f_2 \Leftrightarrow g_1 \text{ und } g_2 \text{
  strukturgleiche BDDs}$.

\begin{algequ}
\texttt{Equiv}(a,b)
\begin{minipage}{0.9\textwidth}
\begin{empheq}[box=\fbox]{align*}
&\text{if a 1-Senke then return \glqq b 1-Senke\grqq}\\
&\text{else if a 0-Senke then return \glqq b 0-Senke\grqq}\\
&\text{else if b Senke then return false}\\
&\text{else if }var(a) = var(b) \text{ then return } \texttt{Equiv}(\underline{then}(a),
 \underline{then}(b)) \wedge \texttt{Equiv}(\underline{else}(a),
 \underline{else}(b))\\
&\text{else return false}
\end{empheq}
\end{minipage}
\end{algequ}

\textbf{Notwendig zur Effizienzsteigerung: Computed
    Table}\\
Computed Table CT mit Eintr\"agen $((a,b,),r)$ als Paare, wobei $a, b$
BDD-Knoten sind und $r$ ein Wahrheitswert ist. Dabei gilt:
\begin{align*}
  ((a,b) ,r) \text{ in CT } \Rightarrow \texttt{Equiv}(a,b) = r
\end{align*}
Die Tabellenoperationen umfassen dabei folgende Funktionalit\"aten:
\begin{itemize}
\item Einf\"ugen eines Paares
\item Testen, ob zu $(a,b)$ ein Eintrag existiert
\item Auslesen des Wahrheitwertes $r$ zum Eintrag $((a,b), r)$
\end{itemize}
Die Implementierung von CTs erfolgt zumeist durch
\href{http://de.wikipedia.org/wiki/AVL-Baum}{AVL-B\"aume}, wodurch
logarithmische Zugriffszeiten m\"oglich sind.

\subsection{Negation}
Gegeben sei eine Funktion \bolf und die Negation $\bar f :
\mathds{B}^n \rightarrow \mathds{B}, \bar
f(x_1,\dots,x_n)=\overline{f(x_1,\dots,x_n)}$.\\
Problem: Ist $g$ OBDD zur Darstellung von $f$, wie bekommt man OBDD
zur Darstellung von $\bar f$?\\
Bei ROBDDS und QOBDDs wird mit $\bar g$ das BDD bezeichnet, welches
$\bar f$ darstellt. Man erh\"alt also die ROBDD- bzw. QOBDD-Negation
durch die Berechnung von $\bar g$ aus $g$.

\begin{ms}
Sei $g$ das ROBDD (QOBDD), welches \bolf ~darstellt. Ist $\bar g$ das
BDD, welches aus $g$ dadurch entsteht, dass man $\mathds{O}$ und
$\mathds{I}$ vertauscht, so stellt $\bar g$ die Funktion $\bar f$ dar.
\qed
\end{ms}

\begin{mdef}
Es seien $g_1, g_2$ OBDDs mit fester Variablenordnung und $x \in X$,
so dass $\underline{var}(a) <x$ f\"ur alle Knoten $a$ von $g_1$ und
$g_2$ gilt. Dann ist OBDD $\underline{cons}(x,g_1,g_2)$ festgelegt
durch:
\begin{align*}
\underline{var}(\underline{cons}(x,g_1,g_2)) =& ~x\\
\underline{then}(\underline{cons}(x,g_1,g_2)) =& ~g1 \\
\underline{else}(\underline{cons}(x,g_1,g_2)) =& ~g2
\end{align*}
\qed
\end{mdef}

Implementierung:\\

\texttt{Cons}($x,g_1,g_2$)
\begin{minipage}{0.4\textwidth}
\begin{empheq}[box=\fbox]{align*}
&new(a)\\
&a.id := \underline{max}(g_1.id, g_2.id) + 1\\
&a.var := x\\
&a.then := g1\\
&a.else := g2 \\
&return ~\texttt{Reduce}(a)
\end{empheq}
\end{minipage}

\begin{mex}

$X = \{x,y,z\}, \pi=(x,y,z)$ $f_1(x,y,z) = y+z, f_2(x,y,z) = y \cdot
z)$
\begin{itemize}
\item[(1)]  Konstruiere $\underline{cons}(x,g_1,g_2)$.\footnote{$f(x,y,z) = x
  \cdot y +x \cdot \bar y \cdot z = \bar x \cdot y \cdot z = x \cdot
  f_1 + \bar x \cdot f_2$}
\item[(2)] 
Konstruiere $\underline{cons}(x,g_1,g_1)=g_1$.
\end{itemize}
\end{mex}

\begin{ms}
Seien $f_1,f_2:\mathds{B}^n \rightarrow \mathds{B}$ dargestellt durch
OBDDs $g_1$ bzw. $g_2$. Dann wird durch $\underline{cons}(x,g_1,g_2)$
die Funktion \bolf ~dargestellt, wobei gilt:
\begin{equation*}
f = x \cdot f_1 + \bar x \cdot f_2
\end{equation*}
\qed
\end{ms}

\begin{algneg}
Voraussetzung: Computed Table $CT$ mit Eintr\"agen $(a,r)$, wobei $a$
und $r$ OBDD-Knoten sind.\\

\texttt{Neg}($a$)
\begin{minipage}{0.6\textwidth}
\begin{empheq}[box=\fbox]{align*}
&\text{if a = 1-Senke then return } \mathds{O}\\
&\text{else if a = O-Senke then return } \mathds{I}\\
&\text{else if } \exists r:(a,r)\text{ in CT then return r mit } (a,r)
\text{ in CT}\\
&\text{else } r_1:= \texttt{Neg}(\underline{then}(a))\\
&\hspace{0.75cm} r_2:= \texttt{Neg}(\underline{else}(a))\\
&\hspace{0.75cm} r~:= \underline{cons}(\underline{var}(a),r_1,r_2)\\
& \hspace{0.75cm} \underline{add}(a,r) \text{ to CT}\\
& \hspace{0.75cm} \text{return r}
\end{empheq}
\end{minipage}
\end{algneg}

\begin{mex}
Sei $X=\{y,z\}$ und $f=y+z$. Konstruiere
$\underline{cons}(y,\mathds{I},\underline{cons}(z,\mathds{I},\mathds{O}))$.\\
Wie sieht die Negation aus?
\end{mex}

\subsection{Bin\"are Operationen}
Es gibt 16 Funktionen von $\mathds{B} \times \mathds{B}$ nach
$\mathds{B}$, da $\vert \mathds{B}^{\mathds{B} \times \mathds{B}}\vert
= 2^4 = 16$. So sind bspw. $+, \cdot, \rightarrow, \leftrightarrow$
solche Funktionen.\\

Wenn \bolf, $g: \mathds{B}^n \rightarrow \mathds{B}, \otimes:
\mathds{B} \times \mathds{B} \rightarrow \mathds{B}$, dann ist $f
\otimes g: \mathds{B}^n \rightarrow \mathds{B}$, wobei $f \otimes g$
definiert ist durch:
\begin{equation*}
(f \otimes g)(x_1,\dots,x_n) = f(x_1,\otimes,x_n) \otimes g(x_1,\dots,x_n)
\end{equation*}
Also $\otimes :  \mathds{B}_n \times \mathds{B}_n \rightarrow
\mathds{B}_n$.\\

Beispiele:
\begin{align*}
+:& ~f + g: \mathds{B}^n \rightarrow \mathds{B} \hspace{0.5cm} (f
\sqcup g)\\
\cdot:& ~f \cdot g: \mathds{B}^n \rightarrow \mathds{B} \hspace{0.5cm} (f
\sqcap g)
\end{align*}

Problem:\\
Gegeben seien Funktionen $f_1,f_2: \mathds{B}^n \rightarrow \mathds{B},
\otimes: \mathds{B} \times \mathds{B} \rightarrow \mathds{B}$ und
OBDDs $g_1$ f\"ur $f_1$ und $g_2$ f\"ur $f_2$.\\
Konstruiere $g_1$ und $g_2$ OBDD zur Darstellung von $f_1 \otimes f_2$
(\underline{Bin\"are Synthese}).\\

Im Folgenden sollen zwei Ans\"atze zur Bin\"aren Synthese betrachtet
werden.
\begin{itemize}
\item[(1)] ITH-Funktion (if-then-else)
\item[(2)] Apply-Funktion
\end{itemize}

Zu (1): ITH-Funktion\\
$ITH: \mathds{B}_n \times \mathds{B}_n \times \mathds{B}_n \rightarrow
\mathds{B}_n, ~ITH(f,g,h) = f \cdot g + \bar f h$\\

Nun soll gezeigt werden, dass die Funktionen $f_1+f_2, f_1 \cdot f_2,
\bar f$ mit der ITH-Funktion darstellbar sind.
\begin{align*}
ITH(f,\0, \1) =&~ f \cdot \0 + \bar f \cdot \1 = \bar f \\
ITH(f_1,f_2, \0 ) =&~ f_1 \cdot f_2 + \overline{f_1} \cdot \0 = f_1 \cdot
f_2\\
ITH(f_1,\1,f_2) = &~ f_1 \cdot \1 + \overline{f_1} \cdot f_2 = f_1 +
\overline{f_1} \cdot f_2 = (f_1 + \overline{f_1}) \cdot (f_1 + f_2) = f_1 + f_2
\end{align*}
Damit ist jede Funktion \bolf ~durch einen Term in ITH und $\0,\1$
ausdr\"uckbar.\\

Zu (2): Apply-Funktion\\
$Apply: \mathds{B}_n \times \mathds{B}_n \times \mathds{B}_2
\rightarrow \mathds{B}_n$, wobei die ersten zwei Argumente, sowie das
Resultat OBDDs sind und das dritte Argument ein Operation ($\otimes$)
ist.\\
Die Apply-Funktion soll nun im weiteren Verlauf verwendet werden.

\begin{ms}
Es sei $f,g: \bol$ und $\otimes: \mathds{B} \times \mathds{B}
\rightarrow \mathds{B}$. Dann gilt f\"ur jede Variable $x_i$
(aufgefasst als Projektion) $x_i: \mathds{B}_n \rightarrow \mathds{B}$: 
\begin{align*}
f \otimes g = x_i \cdot (f_{x_i} \otimes g_{x_i}) + \overline{x_i}
\cdot (f_{\overline{x_i}} \otimes g_{\overline{x_i}})
\end{align*}
\end{ms}

\begin{bw}
Seien $(y_1, \dots, y_n) \in \mathds{B}^n$. Dann gilt $f \otimes
g(y_1,\dots,y_n) = f(y_1,\dots,y_n) \otimes g(y_1,\dots,y_n)$.\\

Rechte Seite:\\
\begin{align*}
&(x_i \cdot (f_{x_i} \otimes g_{x_i}) + \overline{x_i} \cdot
(f_{\overline{x_i}} \otimes g_{\overline{x_i}}))(y_1,\dots,y_n) \\
&= x_i \cdot (f_{x_i} \otimes g_{x_i})(y_1,\dots,y_n) + \overline{x_i} \cdot
(f_{\overline{x_i}} \otimes g_{\overline{x_i}})(y_1,\dots,y_n)\\
&= x_i(y_1,\dots,y_n) \cdot (f_{x_i}(y_1,\dots,y_n) \otimes g_{x_i}(y_1,\dots,y_n)) +
\overline{x_i}(y_1,\dots,y_n) \cdot (f_{\overline{x_i}}(y_1,\dots,y_n) \otimes
g_{\overline{x_i}}(y_1,\dots,y_n))\\
&= y_i \cdot f(y_1,\dots,y_{i-1},1,y_{i+1}\dots,y_n) \otimes
g(y_1,\dots,y_{i-1},1,y_{i+1}\dots,y_n) 
\\&\hspace{0.4cm} + \overline{y_i} \cdot
f(y_1,\dots,y_{i-1},0,y_{i+1},\dots,y_n) \otimes
g(y_1,\dots,y_{i-1},0,y_{i+1},\dots,y_n)\\
&\text{Fall 1: } y_i = 1\\
&= f(y_1,\dots,y_{i-1},1,y_{i+1},\dots,y_n) \otimes
g(y_1,\dots,y_{i-1},1,y_{i+1},\dots,y_n) + 0\\
&= f(y_1,\dots,y_n) \otimes g(y_1,\dots,y_n)\\
&\text{Fall 2: } y_1 = 0\\
&= 0 + f(y_1,\dots,y_{i-1},1,y_{i+1},\dots,y_n) \otimes
g(y_1,\dots,y_{i-1},1,y_{i+1},\dots,y_n)\\
&= f(y_1,\dots,y_n) \otimes g(y_1,\dots,y_n)\\
\end{align*}
\qed
\end{bw}

\begin{binsys}
Voraussetzung: $f_1,f_2: \bol$ werden durch die QOBDDs $g_1$
bzw. $g_2$ (mit gleicher Variablenordnung) dargestellt und $\otimes:
\mathds{B} \rightarrow \mathds{B}$ sei die gegebene bin\"are
Operation.\\
Dann berechnet \texttt{Apply}($g_1,g_2,\otimes$) das QOBDD f\"ur $g_1
\otimes g_2$ (mit CT).\\

\texttt{Apply}($a,b,\otimes$)
\begin{minipage}{0.8\textwidth}
  \begin{empheq}[box=\fbox]{align*}
    \text{if } a \wedge b \text{ Senken }&\text{then return } a \otimes b
    \text{ als QOBDD}\\ 
    &\text{elseif } \exists r: ((a,b),r) \text{ in CT then return } r \text{ mit } ((a,b),r) \text{ in CT}\\
    &\text{else } r_1 := \texttt{Apply}(\underline{then}(a),
    \underline{then}(b), \otimes)\\
    &\hspace{0.7cm}r_2 := \texttt{Apply}(\underline{else}(a), \underline{else}(b),
    \otimes)\\
    &\hspace{0.7cm}r := \texttt{Cons}(\underline{var}(a), r_1, r_2)\\
    &\hspace{0.7cm}\underline{add}((a,b),r) \text{ to CT}\\
    &\text{return } r
  \end{empheq}
\end{minipage}
\end{binsys}

\begin{mex}
(1) Seien $x$ und $f_1,f_2: \mathds{B}^n \rightarrow \mathds{B}, \otimes:
\mathds{B} \rightarrow \mathds{B}$ gegeben mit 
\begin{align*}
&f_1(x) = x, ~ f_2(x) = \bar x, ~ \otimes = \cdot\\
&(f_1 \cdot f_2)(x) = x \cdot \bar x = 0
\end{align*}
($\texttt{Apply}(g_1,g_2,\otimes)$ durchf\"uhren, wobei $g_1$ und $g_2$
  QOBDDs zu $f_1$ bzw. $f_2$.) \\

(2) $X={x,y}, ~\pi=(x,y), ~f_1: \mathds{B}^2 \rightarrow \mathds{B},
~f_2: \mathds{B}^2 \rightarrow \mathds{B}, ~\otimes: \mathds{B}
\rightarrow \mathds{B}$.
\begin{align*}
&f_1(x,y) = x\\
&f_2(x,y) = x+y\\
&\otimes = \cdot\\
&(f_1 \cdot f_2)(x,y) = f_1(x,y) \cdot f_2(x,y) = x \cdot (x+y) = x
\end{align*}
\end{mex}

\begin{apply-robdd}
Im Gegensatz zu den QOBBDs m\"ussen hier 4 F\"alle unterschieden
werden.
\begin{itemize}
\item[(1)] $a$ und $b$ sind Senken:\\ $\texttt{Apply}(a,b,\otimes) = a
  \otimes b$ (als ROBDD)
\item[(2)] $a$ und $b$ sind keine Senken und
  $\underline{var}(a)=\underline{var}(b)=x$:\\
$\texttt{Apply}(a,b,\otimes) =\texttt{Cons}(x,
\texttt{Apply}(\underline{then}(a), \underline{then}(b), \otimes),
\texttt{Apply}(\underline{else}(a), \underline{else}(b), \otimes))$
\item[(3)] $a$ ist keine Senke und $b$ beliebig, aber
  $\underline{var}(b) < \underline{var}(a) = x$, falls $b$ keine
  Senke ist:\\
$\texttt{Apply}(a,b,\otimes)=\texttt{Cons}(x,
\texttt{Apply}(\underline{then}(a), b, \otimes),
\texttt{Apply}(\underline{else}(a),b,\otimes))$
\item[(4)] $b$ keine Senke und $a$ beliebig, aber $\var(a) < \var(b)=x$,
  falls $a$ keine Senke ist:\\
$\texttt{Apply}(a,b,\otimes)=\texttt{Cons}(x,
\texttt{Apply}(a,\then(b), \otimes), \texttt{Apply}(a, \el(b), \otimes))$
\end{itemize}
\end{apply-robdd}

\begin{mex}
Seien $X=\{x\}, ~ f_1,f_2: \mathds{B}^2 \rightarrow \mathds{B},
\otimes: \mathds{B} \rightarrow \mathds{B}$.
\begin{align*}
&f_1(x,y) = x\\
&f_2(x,y) =  x + y\\
&\otimes = \cdot\\
&(f_1 \otimes f_2)(x,y) = x \cdot (x+y) = x
\end{align*}
\end{mex}

\subsection{BDD-Pakete}
Anwendung:
\begin{itemize}
\item[(1)] Schaltungsentfwurf, VLSI-Design
\item[(2)] \href{http://de.wikipedia.org/wiki/Model_Checking}{Symbolisches Model Checking}
\item[(3)] Algorithmik, gro\ss e Graphen
\item[(4)] Spieltheorie (\href{http://www.informatik.uni-kiel.de/~stb/}{Stefan Bolus})
\item[(5)] Relationentheorie
\end{itemize}

Pakete:
\begin{itemize}
\item[(1)] \href{http://vlsi.colorado.edu/~fabio/CUDD/}{CUDD}
\item[(2)] \href{http://www.cs.cmu.edu/~modelcheck/bdd.html}{CMU-BDD-Paket}
\item[(3)] \href{http://buddy.sourceforge.net/manual/main.html}{BuDDy}
\item[(4)] \href{http://www.sosy-lab.org/~dbeyer/CrocoPat/}{CrocoPat}
\item[(5)] \href{http://www.informatik.uni-kiel.de/~progsys/relview/}{RelView} und \href{http://korsika.informatik.uni-kiel.de/kure2/}{Kure}
\end{itemize}


%%%%%%%%%%%%%%%%%%%%
%%%%%%%%%%%%%%%%%%%%
\newpage
%%%%%%%%%%%%%%%%%%%%
%%%%%%%%%%%%%%%%%%%%

\section{Anwendung der Spieltheorie}
In diesem Abschnitt werden folgende Themen bearbeitet:
\begin{itemize}
\item Einfache Spiele und QOBDDs
\item Berechnung von Schl\"usselspielern
\item Bestimmung der W\"unschenswert-Relation
\end{itemize}

Kurz zur Spieltheorie:\\
Man unterscheidet zwischen konkurrierenden und kooperierenden
Spielen. Zweiteres wird noch in einfache Spiele\footnote{Nutzen ist
  hier 0 oder 1} und nicht-einfach Spiele unterteilt.\\
Mit BDDs kann man einfache Spiele besonders gut berechnen. Einfache
Spiele entsprechen monotonen Funktionen.

\subsection{Einfache Spiele und QOBDDs}
\begin{mdef}
Ein einfaches Spiel ist ein Paar $(X,\mathcal{W})$ mit:
\begin{itemize}
\item[(1)] $X$ endliche, nicht-leere Menge von Spielern
\item[(2)] $\mathcal{W} \subseteq 2^X$ mit der Eigenschaft, dass f\"ur
  alle $Y,Z \in 2^X$:
\begin{align*}
Y \in \mathcal{W} \wedge Y \subseteq Z \Rightarrow Z \in W
\end{align*}
$Y \in 2^X$ hei\ss t \underline{Koalition}, $Y \in \mathcal{W}$ hei\ss
t \underline{gewinnende Koalition} und $Y \in 2^{X} \setminus
  \mathcal{W}$ \underline{verlierende Koalition}.
\qed
\end{itemize}
\end{mdef}

\begin{mdef}
Sei $(X, \mathcal{W})$ ein einfacher Spiel. Ein Paar $(Q, w)$ mit $Q
\in \mathds{N}_{>0}$ und $w: X \rightarrow \mathds{N}$ hei\ss t \underline{gewichtete
  Dartellung}, falls f\"ur alle $Y \in 2^X$ gilt:
\begin{align*}
Y \in \mathcal{W} \Leftrightarrow \sum_{x \in Y} w(x) \geq Q
\end{align*}
Besitzt ein Spiel eine gewichtete Darstellung, so hei\ss t es
\underline{gewichtetes Mehrheitsspiel} (GMS).
\qed
\end{mdef}
Normalfall der Schreibweise ist dabei wie folgt:\\
$X=\{x_1,\dots, x_n\}, \lbrack Q; w_1, \dots, w_n \rbrack$, wobei $w_i
= w(x_i)$ und $Q$ Quote.

\begin{mex}\label{mex:413}
(1) Sei $(X,\mathcal{W})$ mit $X = \{x_1,x_2, x_3\}, ~ \mathcal{W} =
\{\{x_1,x_2\}, \{x_2,x_3\}, \{x_1,x_3\}.\{x_1,x_2,x_3\}\}$.\\
$(X,\mathcal{W})$ ist GMS mit $\lbrack 6;5,5,1\rbrack$.\\
Warum sind die Mengen in $\mathcal{W}$ gewinnende Koalitionen?\\

(2) Sei $(X,\mathcal{W})$ einfaches Spiel, gegeben durch $X = \{ x_1, x_2,
x_2, x_4, x_5\}$ und $\lbrack 312; 239, 146, 93, 76, 68
\rbrack$\footnote{oder $\lbrack 3; 2, 1, 1, 1, 0 \rbrack$ in der
  minimalen Variante}. Hierbei stehen die Elemente aus $X$ f\"ur die
folgenden Partein:\\
$x_1 \mathrel{\widehat{=}}$  CDU/CSU, $x_2 \mathrel{\widehat{=}}$
SPD, $x_3 \mathrel{\widehat{=}}$  = FDP, $x_4 \mathrel{\widehat{=}}$
Die Linke, $x_5 \mathrel{\widehat{=}}$  Die Gr\"unen und die Gewichte
entsprechen der Sitzverteilung nach der Bundestagswahl 2009.\\
Wie sieht nun $\mathcal{W}$ aus?
\end{mex}

\begin{mdef}
Sei \sg einfaches Spiel. Dann hei\ss t dies
\underline{Vektorgerichtetes Mehrheitsspiel} (VGMS), falls es $k \in
\mathds{N}_{>0}$ und $Q^{(1)},\dots,Q^{(k)} \in \mathds{N}_{>0}$ und
$w^{(j)}: X \rightarrow \mathds{N}$ f\"ur alle $j \in \{1, \dots, k\}$
gibt mit der Eigenschaft, dass f\"ur alle $Y \in 2^X$ gilt:
\begin{align*}
Y \in \mathcal{W} \Leftrightarrow \forall j \in \{1, \dots, k\}:
\sum_{x \in Y} w^{(j)}(x) \geq Q^{(j)}
\end{align*}
\qed
\end{mdef}

$(Q^{(j)},w^{(j)})$ hei\ss t $j$-tes GMS von VGMS. Weiter gilt:
\begin{align*}
Y \text{ gewinnend in VGMS } \Leftrightarrow Y \text{ gewinnend in
  jedem der einzelnen GMS}
\end{align*}
Normfall der Schreibweise ist dabei wie folgt:\\
$\lbrack Q^{(1)},w^{(1)}_1,\dots, w^{(1)}_n \rbrack \wedge \dots
\wedge \lbrack Q^{(k)},w^{(k)}_1,\dots, w^{(k)}_n \rbrack$ mit
$w_i^{(j)}=w^{(j)}(x_i)$ und $x_1, \dots, x_n$ als Reihenfolge.

\begin{mex}\label{mex:415}
Vertrag von Nizza\\
$X = \{x_1, \dots, x_{27}\}$\\
$\lbrack 265; 29, 29, 29, 29, 27, 27, 14, 13, 12, 12, 12, 12, 12, 10,
10, 10, 7, 7, 7, 7, 7, 4, 4, 4, 4, 4, 3\rbrack ~\wedge~ \lbrack 14; 1,
\dots, 1 \rbrack ~\wedge$\\
$\lbrack 620; 170, 123, 122, 120, 82, 89, 47, 33, 22, 21, 21, 21, 21,
18, 17, 17, 11, 11, 11, 8, 8, 4, 4, 3, 2, 1, 1\rbrack$\\
VGMS kann dabei nicht auf GMS minimiert bzw. zur\"urckgef\"uhrt werden.
\end{mex}

\begin{mk}
Bei einem einfachen Spiel \sg ist im Folgenden $X = \{x_1, \dots, x_n\}$
und $X$ ist auch die Menge der Variablen zur Definition von \bolf und
der darstellenden QOBDDs Variablenordnung sei $\pi = (x_1, \dots, x_n)$.
\qed
\end{mk}

\begin{mdef}
Sei \sg einfaches Spiel. Zu $Y \in 2^X$ ist der
\underline{charakteristische Verktor} $\chi(Y) \in \mathds{B}^n$
definiert durch:
\begin{align*}
\chi(Y)_i =1 \Leftrightarrow x_i \in Y \text{ f\"ur alle } i \in \{1,
\dots, n\}
\end{align*}
\qed
\end{mdef}

\begin{mdef}
Sei \sg einfaches Spiel. 
\begin{itemize}
\item[(1)] Eine Funktion \bolf stellt \sg dar, falls f\"ur alle $Y \in
  2^X$ gilt:
\begin{align*}
Y \in W \Leftrightarrow f(\chi(Y)) = 1
\end{align*}
\item[(2)] Ein QOBDD $g$ \"uber $X$ stellt \sg dar, falls es die
  Funktion $f$ darstellt, die auch \sg darstellt.
\end{itemize}
\qed
\end{mdef}

\begin{mex}
(1) Spiel von \hyperref[mex:413]{Bsp. \ref{mex:413} (1)}\\
(2) \hyperref[mex:413]{Bundestag Sept. 2009}
\end{mex}

\begin{mdef}
Sei $g$ ein QOBDD und $w = (p_1, \dots, p_k)$ ein Weg von einem
inneren Knoten in $g$ zu $\mathds{I}$. Dann definiert man
\begin{align*}
\underline{eins}(w) = \{\var(\alpha(p_i)) \mid 1 \leq i \leq k \wedge
p_i \text{ 1-Pfeil}\}
\end{align*}
\qed
\end{mdef}

\begin{ms}
Es sei \sg ein einfaches Spiel und $r$ Wurzel des QOBDDs, das \sg
darstellt. Dann gilt f\"ur alle $Y \in 2^X$:
\begin{align*}
Y \in W \Leftrightarrow \text{ Es gibt Weg } w \text{ von } r \text{
  nach } \mathds{I} \text{ mit } Y = \underline{eins}(w)
\end{align*}
\qed
\end{ms}

\begin{mdef}
Es sei ein QOBDD $g$ mit Knotenmengen $V$ gegeben. Dann ist die
Funktion $\underline{set}:V \rightarrow 2^X$ definiert durch:
\begin{align*}
  \set(a) = \begin{cases} \{\eins(w) \mid w \text{ ist Weg von } a
    \text{ nach } \mathds{I} \} &,~ a \text{ ist innerer Knoten }\\
    \{\emptyset\} &,~ a = \mathds{I}\\
    ~\emptyset &,~ a = \mathds{O}
  \end{cases}
\end{align*}
\qed
\end{mdef}

\begin{ms}
Es seien $g$ QOBDD und $a$ innerer Knoten. Dann gilt:
\begin{align*}
\set(a) = \{Y+\var(a) ~\vert~ Y \in \set(\then(a))\} \cup \set(\el(a))
\end{align*}
\end{ms}

\begin{bw}
\begin{align*}
\set(a) =&~ \{ eins(w) ~\vert~ w \text{ Weg von } a \text{ nach }
\mathds{I}\}\\
=&~ \{ eins(w) ~\vert~ w \text{ Weg von } a \text{ nach } \mathds{I}
~\wedge~ w_1 \text{ 1-Pfeil}\} \cup \{eins(w) ~\vert~ w \text{ Weg von } a \text{ nach } \mathds{I}
~\wedge~ w_1 \text{ 0-Pfeil}\} \\
=&~ \{ eins(w)+\var(a) ~\vert~ w \text{ Weg von } \then(a) \text{ nach
} \mathds{I}\} \cup \{ eins(w) ~\vert~ w \text{ Weg von } \el(a)
  \text{ nach } \mathds{I}\} \\
=&~ \{ Y+\var(a) ~\vert~ \exists\text{ Weg } w \text{ von } \then(a) \text{ nach }
\mathds{I} \text{ mit } Y = eins(w)\} \cup \set(\el(a))\\
=&~\{Y+\var(a) ~\vert~ Y \in \set(\then(a)\} \cup \set(\el(a))
\end{align*}
\qed
\end{bw}

Wenn \sg ein einfaches Spiel ist und $r$ die Wurzel des dargestellten QOBDDs,
dann gilt $\set(r) = \mathcal{W}$.

\begin{ms}\label{sat:fiq}
Es sei \sg ein GMS mit gewichteter Darstellung \gms. Definiere f\"ur
alle $i \in \{1, \dots, n+1\}$ und $q \in \mathds{R}$ (die Funktion
$f:\{1, \dots, n+1\} \times \mathds{R} \rightarrow QOBDD$):
\begin{align*}
&f(i,q) = \begin{cases}
\mathds{I} &,~ i = n+1 \wedge q \leq 0\\
\mathds{O} &,~ i = n+1 \wedge q > 0\\
\cons(x_i, f(i+1, q-w_i), f(i+1, q)) &,\text{ sonst }
\end{cases}\\
&\text{Dann gilt:}\\
&\set(f(i,q)) = ~\biggr \{ Y \in 2^{\{x_i,\dots,x_n\}} \biggr \vert \sum_{x \in Y} w(x)
\geq q \biggr\}\\
&\text{Insbesondere gilt:}\\
&\set(f(1,Q))= ~\biggr\{Y \in 2^X \biggr \vert \sum_{x \in Y} w(x) \geq
Q \biggr \} = \mathcal{W}
\end{align*}
\end{ms}

\begin{bw}
Induktion nach Differenz von $n+1-i$\\
\underline{IA:} $n+1-i = 0$, d.h. $i = n+1$\\
Fall 1. $q \leq 0$. Dann gilt 
\begin{align*}
&~f(i,q) = \mathds{I}\\
&\biggr \{Y \in 2^{\emptyset} \biggr \vert \sum_{x \in Y} w(x) \geq q
\biggr \} = \{\emptyset\} = \set(\mathds{I}) = \set(f(i,q))
\end{align*}
Fall 2. $q > 0$. Dann gilt
\begin{align*}
&~f(i,q) = \mathds{O}\\
&\biggr \{Y \in 2^{\emptyset} \biggr \vert \sum_{x \in Y} w(x) \geq q
\biggr \} = \emptyset = \set(\mathds{O}) = \set(f(i,q))
\end{align*}

\underline{IS:} Sei $n+1-i>0$, d.h. $i < n+1$. Dann gilt f\"ur alle $Y
\in 2^X$
\begin{align*}
  Y \in \set(f(i,q)) \Leftrightarrow &~ Y \in
  \set(\cons(x_i,f(i+1,q-w_i),
  f(i+1, q))\\
  \Leftrightarrow &~ (Y \subseteq \{x_i, \dots, x_n\} \wedge x_i \in Y
  \wedge \sum_{x \in Y-x_i} w(x) \geq q - w_i) \\ &~\vee (Y \subseteq
  \{x_i, \dots, x_n\} \wedge x_i \in Y
  \wedge \sum_{x \in Y-x_i} w(x) \geq q)\\
  \Leftrightarrow &~ (Y \subseteq \{x_i, \dots, x_n\} \wedge x_i \in Y
  \wedge \sum_{x \in Y} w(x) \geq q)\\
  & ~\vee (Y \subseteq \{x_i, \dots, x_n\} \wedge x_i \in Y \wedge
  \sum_{x \in Y} w(x) \geq q)\\
  & \Leftrightarrow Y \subseteq \{x_1, \dots, x_n\} \wedge x_i \in Y
  \wedge \sum_{x \in Y} \leq q
\end{align*}
\qed
\end{bw}

\begin{mex}
Sei $X = \{x_1,x_2,x_3\}$ gewichtete Darstellung $\lbrack 6;
5,5,1\rbrack$. Bestimme $f(1,6)$.
\end{mex}

\subsection{Bestimmung von Schl\"usselspielern}
\begin{mdef}
Sei \sg einfaches Spiel.
\begin{itemize}
\item[(1)] Eine gewinnende Koalition $Y \in \mathcal{W}$ hei\ss t
  \underline{minimal} gewinnend, falls $Z \not \in \mathcal{W}$ f\"ur
  alle $Z \subseteq Y$.
\item[(2)] $\mathcal{W}_{min} = \{ Y \in W ~\vert~ Y \text{ minimal
    gewinnend} \}$
\end{itemize}
\qed
\end{mdef}

\begin{mdef}
Sei \sg einfaches Spiel. Ein Spieler $x \in X$ hei\ss t
\begin{itemize}
\item[(1)] \underline{Diktator}, falls $\mathcal{W}_{min} =
\{\{x\}\}$,
\item[(2)] \underline{Vetospieler}, falls f\"ur alle $Y \in
  \mathcal{W}$ gilt: $x \in Y$,
\item[(3)] \underline{belanglos}, falls f\"ur alle $Y \in
  \mathcal{W}_{min}$ gilt: $x \not \in Y$.
\end{itemize}
\qed
\end{mdef}

Der Diktator ist am m\"achtigsten; Vetospieler k\"onnen nichts
erzwingen, aber alles verhindern; belanglose Spieler haben keinerlei
Macht.

\begin{ms}
In einem einfachen Spiel gibt es h\"ochstens einen Diktator.
\qed
\end{ms}

\begin{ms}
Ist $x \in X$ ein Diktator im einfachen Spiel \sg, so gilt f\"ur alle
$y \in X \setminus \{x\}: y \text{ ist belanglos}$.
\end{ms}

\begin{bw}
Da $x$ Diktator ist, gilt $\mathcal{W}_{min} = \{\{x\}\}$. Ist $Y \in
\mathcal{W}_{min}$, so gilt $Y = \{\{x\}\}$, also gilt $y
\not \in Y$, da $y \not = x$. Damit ist $y$ belanglos.

\qed
\end{bw}

\begin{mex}
(1) Im Spiel, dass den Bundestag von 1957 modelliert ist $x_1$
Diktator. Dabei sei $X =\{x_1,x_2,x_3,x_4\}$ mit gewichteter
Darstellung $\lbrack 260; 277,181, 44, 17 \rbrack$.\\

(2) Sei $X = \{x_1, \dots, x_15\}$ mit gewichteter Darstellung
$\lbrack 39; 7, 7, 7, 7, 7, 1 ,1 ,1 ,1 ,1, 1, 1, 1, 1, 1
\rbrack$. Dieses Spiel modelliert den UN Sicherheitsrat. Die
Darstellung als VGMS mit 2 GMS sieht dabei wie folgt aus:
$\lbrack 5; 1, 1, 1, 1, 1, 0, 0, 0, 0, 0, 0, 0, 0, 0, 0, 0 \rbrack
~\wedge~ \lbrack 9; 1, \dots, 1\rbrack$\\

(3) Im Spiel, das den 1. EU Vetrag von 1958 modelliert, ist $x_6$
(Luxemburg) ein belangloser Spieler. Dabei sei $X = \{x_1,\dots,x_6\}$
GMS mit gewichteter Darstellung $\lbrack 12; 4, 4, 4, 2, 2, 1\rbrack$.
\end{mex}

\begin{ml}
Es sei \sg einfach Spiel. Dann gilt f\"ur alle $x \in X$:
\begin{align*}
x \text{ belanglos } \Leftrightarrow \forall~ Y \in \mathcal{W}: Y-x
\in \mathcal{W}
\end{align*}
\end{ml}

\begin{bw}
\glqq $\Rightarrow$\grqq: Sei $Y \in \W$ beliebig. Der Fall, dass $x
\not \in Y$ ist klar, daher sei $x \in Y$. Definiere:
\begin{align*}
\mathcal{K}:= \{ Z \in \W ~\vert~ x \in Z \wedge Z \subseteq Y \}
\end{align*}
$\mathcal{K} \not = \emptyset$, da $Y \in \mathcal{K}$. Wegen $\vert X
\vert < \infty$ gilt auch $\mathcal{K} < \infty$ und damit existiert
in $\mathcal{K}$ mindestens ein Element $Z_0$.

Angenommen $Y-x \not \in \W$. Dann gilt $Z_0-x \not \in \W$, denn
$Z_0-x \in \W$ w\"urde $Y-x \in \W$ implizieren (Monotonie). Also gilt
$Z_0$ minimal gewinnend mit $x \in Z_0$. Das ist ein Widerspruch sazu,
dass $x$ belanglos ist.\\
\glqq $\Leftarrow$\grqq: Sei $x$ nicht belanglos. Dann gibt es $Z \in
\W_{min}$ mit $x \in Z$. Da $Z \in \W_{min}$ folgt $Z \in \W$. Also
gilt nach Annahme $Z-x \in \W$ und damit wiederum $Z \not \in
\W_{min}$. Widerspruch!
\qed
\end{bw}

Folgende Aussagen sind ebenfall \"aquivalent:
\begin{align*}
x \text{ belanglos } \Leftrightarrow &~ \forall~ Y \in 2^X: Y \in \W
\leftrightarrow Y-x \in \W\\
&\text{ Sei nun } f \text{ eine Funktion, die QOBDD darstellt und }
x=x_i. \text{ Dann gilt: }\\
x_i \text{ belanglos } \Leftrightarrow &~ \forall~
y_1,\dots,y_n:~ f(y_1,\dots,y_{i-1},1,y_{i+1},\dots, y_n)=f(y_1,\dots,y_{i-1},0,y_{i+1},\dots,y_n)
\end{align*}

\begin{ms}
Sei \sg durch QOBDD $g$ dargestellt. Dann gilt f\"ur alle $x \in X$:
\begin{align*}
x_i \text{ belanglos } \Leftrightarrow \text{ f\"ur alle } a \in V
\text { mit } \var(a)=x_i \text{ gilt } \then(a) = \el(a)
\end{align*}
\qed
\end{ms}

\begin{ms}
Seien \sg einfach Spiel, $g$ darstellendes QOBDD und $x \in X$. Dann
gilt:
\begin{align*}
x \text{ Vetospieler } \Leftrightarrow \text{ F\"ur alle Knoten } a
\text{ mit } \var(a) = x \text{ gilt } \set(\el(a)) = \emptyset
\end{align*}
\end{ms}

\begin{bw}
Es soll bewiesen werden:
\begin{align*}
x \text{ kein Vetospieler } \Leftrightarrow~ \exists \text{ Knoten } a
\text{ und } \var(a) = x \text{ und } \set(\el(a)) \not = \emptyset
\end{align*}
Zum Beweis sei o.B.d.A. $x = x_i$ angenommen.\\

\glqq $\Rightarrow$\grqq~Sei $x_i$ kein Vetospieler. Dann gibt es
$Y \in W$ mit $x \not \in Y$. Also gibt es einen Weg von der Wurzel
$r$ zu $\mathds{I}$ mit $Y = \eins(w)$. Sei $w =
(p_1,\dots,p_n)$. Dann ist $\alpha(p_i) \not \in Y$ und somit $p_i$
0-Pfeil.\\
Es gilt mit $a:= \alpha(p_i)$, dass $\var(a) = x_i$. Definiere nun $w':=
(p_{i+1}, \dots, p_n)$. Dann ist $w'$ Weg von $\el(a)$ nach
$\mathds{I}$. (Damit gilt $\eins(w) \not = \emptyset$.) Weil $w'$ ein
Weg von $\el(a)$ nach $\mathds{I}$ ist, gilt $\set(\el(a)) \not =
\emptyset$.\\

\glqq $\Leftarrow$\grqq~Es sei $a$ ein Knoten mit $\var(a) = x_i$ und
$\set(\el(a)) \not = \emptyset$. Damit gibt es einen Weg $w'$ von $\el(a)$
nach $\mathds{I}$. Erg\"anze $w'$ zu einem Weg von $r$ nach
$\mathds{I}$ namens $w$. Sei nun $Y := \set(w)$. Dann gilt $Y \in
\W$. Nach Konstruktion von $w$ gilt, dass $w_i$ ein 0-Pfeil ist. Damit
$x_i \not \in \set(w)$, also $x_i \not \in Y$. Also ist $x_i$ kein
Vetospieler. \qed
\end{bw}

\begin{ms}
Es seien \sg einfach Spiel, $g$ darstellendes QOBDD und $x_i$
Spieler. Dann gilt $x_i$ Vetospieler genau dann, wenn f\"ur alle
Knoten der Schicht $i$, $\mathds{I}$ \"uber alle Wege
\"uber den $\el$-Nachfolger nicht erreichbar ist.
\qed
\end{ms}

\begin{ml}
Es sei \sg ein einfaches Spiel. Dann gilt f\"ur alle $x \in X$:
\begin{align*}
x \text{ Diktator } \Leftrightarrow~ \W = \{Y \in 2^X ~\vert~ x \in
Y\}
\end{align*}
\end{ml}

\begin{bw}
\glqq $\Longrightarrow$\grqq ~zu zeigen: $Y \in \W \Leftrightarrow x \in Y$ f\"ur
alle $Y \in 2^X$\\
\glqq $\Rightarrow$\grqq~Wenn $Y \in \W$, dann gibt es $Z \in
\W_{min}$ mit $Z \subseteq Y$ (da $\vert X\vert < \infty$). Da $x$
Diktator ist, gilt $\W_{min} = \{\{x\}\}$, also $Z = \{x\}$, also $x
\in Y$.\\
\glqq $\Leftarrow$\grqq~Sei $x \in Y$. Dann $\{x\} \subseteq Y$. Da
$x$ ein Diktator ist, gilt $\{x\} \in \W_{min} \subseteq W$. Monotonie
und $\{x\} \in Y$ bringen $Y \in \W$.\\

\glqq $\Longleftarrow$\grqq~Es gilt $\{x\} \in \W$, da $\{x\} \in 2^X$ und
$x \in \{x\}$ und vorausgesetzer Gleichheit. Also gilt $\{x\} \in
\W_{min}$. Sei $Y \in \W_{min}$. Dann gelten $Y \in \W$ und $x \in Y$
wegen gefordeter Gleichheit. Also gilt $Y = \{x\}$ und letztendlich
ist damit $\{x\}$ ein Diktator.\\
\qed
\end{bw}

Im Folgenden soll ein spezieller QOBDD namens $\ith$ betrachtet
werden. Daf\"ur ist mit $a$ als Wurzel des
QOBDDs definiert:
\begin{align*}
\set(\ith(i)) = \{ eins(w) ~\vert~ w \text{ Weg von } a \text{ nach }
\mathds{I}\} = \{Y \in 2^X ~\vert~ x_i \in Y\}
\end{align*}

\begin{ms}
Es sei \sg einfaches Spiel und $g$ das darstellende QOBDD. Es ist $x_i
\in X$ genau dann Diktator, wenn $g$ gleich $\ith(i)$ ist.\\
\qed
\end{ms}

BILD von $\ith$.

\subsection{Die W\"unschenswert-Relation}
\begin{mdef}
Sei \sg einfaches Spiel. Die Relation $\preccurlyeq _{I}$ auf $X$
ist f\"ur alle $x,y \in X$ definiert durch:
\begin{align*}
x \ww y \Leftrightarrow &~\forall~ Y \in 2^X: x \not \in Y \wedge y
\not \in Y \wedge Y+x \in \W \Rightarrow Y+y \in \W\\
\Leftrightarrow &~\forall Y \in 2^{X-x-y}: Y+x \in \W \Rightarrow Y+y
\in \W
\end{align*}
$\ww$ hei\ss t \underline{W\"unschenswert-Relation} (auf Spielern) und
$x \ww y$ bedeutet \glqq $y$ ist w\"unschenswerter als $x$\grqq.\\
\qed
\end{mdef}

\begin{ms}
F\"ur alle einfachen Spiele \sg ist $\ww$ eine
Quasi-Ordnung\footnote{also reflexiv und transitiv}.
\end{ms}

\begin{bw}
Reflexivit\"at:\\
trivial\\

Transitivit\"at:\\
Seien $x,y,z \in X$ mit $x \ww y$ und $y \ww z$. zum Beweis von $x \ww
z$ sei $Y \in 2^X$ mit $x \not \in Y$ und $z \not \in Y$ beliebig
vorgegeben. Weiterhin sei $Y+x \in \W$.\\
\underline{Fall 1:} $y \not \in Y$ Wegen $x \not \in Y$, $y \not
\in Y$ und $x \ww y$ sowie $Y+x \in \W$ folgt $Y+x \in \W$. Analog ist
$Y+x \in W$.\\
\underline{Fall 2:} $y \in Y$.\\
Unterfall 1. $x=z$ trivial\\
Unterfall 2. $x \not = z$ Definiere $Y' := Y-y.$ Dann $x,y,z
  \not \in Y'$.
\begin{align*}
Y+x \in \W \Leftrightarrow &~ Y'+y+x \in \W\\
(x \not = y) \Leftrightarrow &~ Y'+x+y \in \W\\
(y \not \in Y'+x, z \not \in Y'+x, y \ww z) \Rightarrow &~ Y'+x+z \in
\W\\
x \not = z \Leftrightarrow &~ Y'+z+x \in \W\\
x \not \in Y'+z, y \not \in Y'+z, x \ww y \Leftrightarrow &~ Y'+z+y
\in \W\\
y \not = z \Leftrightarrow &~ Y'+y+z \in \W\\
\Leftrightarrow &~ Y+z \in \W
\end{align*}
\qed
\end{bw}

\begin{ms}
Ist \sg GMS mit gewichteter Darstellung $(Q,w)$, so gilt f\"ur alle
$x,y \in X$:
\begin{align*}
w(x) \leq w(y) \Rightarrow x \ww y
\end{align*}
\end{ms}

\begin{bw}
Sei $Y \in 2^X$ mit $x \not \in Y$ und $y \not \in Y$. Dann gilt
\begin{align*}
Y+x \in \W \Leftrightarrow &~ w(x) + \sum_{z \in Y} w(z) \geq Q\\
\Rightarrow &~ w(y) + \sum_{z \in Y} w(z) \geq Q\\
\Leftrightarrow &~ Y+y \in \W
\end{align*}
\qed
\end{bw}

\begin{ms}
Die Relation $\ww$ und $\approx_I$ auf $X$ sind f\"ur ein einfach
Spiel \sg und alle Spieler $x,y \in X$ definiert durch:
\begin{align*}
x \prec_I y &\Leftrightarrow (x \ww y) ~\wedge~ \neg(y \ww x)\\
& \text{\glqq y echt w\"unschenswerter als x\grqq} 
\end{align*}
\begin{align*}
x \approx_I y &\Leftrightarrow (x \ww y) ~\wedge~ (y \ww x)\\
& \text{\glqq x,y gleich w\"unschenswert\grqq} 
\end{align*}
Dabei gilt, dass $\prec_I$ eine strikte
Quasi-Ordnung\footnote{asymmetrisch und transitiv bzw. irreflexiv und
  transitiv} und $\approx_I$ eine
\"Aquivalenzrelation\footnote{symmetrisch, transitiv und reflexiv}
ist.
\qed
\end{ms}

\begin{ml}
Es sei \sg ein einfaches Spiel, dann sind f\"ur alle $x,y \in X$ mit
$x \not = y$ die folgenden zwei Aussagen \"aquivalent:
\begin{align*}
&\text{ (1) } x \ww y\\
&\text{ (2) } \{Y-x ~\vert~ x \in Y \wedge Y \in \W \wedge y \not \in
Y\} \subseteq \{Y-y ~\vert~ y \in Y \wedge Y \in \W\}
\end{align*}
\end{ml}

\begin{bw}
\glqq(1) $\Rightarrow$ (2)\grqq~ Es gelte $x \ww y$. Sei $Z \in \{Y-x ~\vert~ x \in Y \wedge Y \in \W \wedge y \not \in
Y\}$. Also gibt es $Y$ mit $Y \in \W$,
$x \in Y$, $y \not \in Y$ und $Z = Y-x$. 
\begin{align*}
Y \in W \Leftrightarrow &~ Y-x+x \in W \hspace{2cm}&\text{ wegen } x \in
Y \text{ gilt } Y=Y-x+x\\
\Leftrightarrow &~ Z+x \in \W \\
\Rightarrow &~ Z+y \in \W &\text{ wegen } x \ww y, x \not
\in Z \text{ und } y \not \in Z
\end{align*}
Definiere nun $Y' := Z+y$. Dann gilt $Y' \in \W$. Weiter gilt $y \in
Y'$. Also ist $Y'-y \in  \{Y-y ~\vert~ y \in Y \wedge Y \in
\W\}$. Weiter ist $y \not \in Y$, also $Y'-y=Z+y-y=Z$ und letztendlich
$Z = Y'-y$. Damit gilt $Z \in  \{Y-y ~\vert~ y \in Y \wedge Y \in
\W\}$.\\

\glqq(2) $\Rightarrow$ (1)\grqq~ Es gelte die Mengeninklusion. Zum
Beweis von $x \ww y$ sei $Y \in 2^X$ mit $x \not \in Y$, $y \not \in
Y$ und $Y+x \in \W$ gegeben.\\
Definiere nun $Z = Y+x$. Dann ist $Z \in \W$, $x \in Z$ und $y \not
\in Z$ (wegen $x \not = y$). Damit gilt $Z-x \in \{Y-x ~\vert~ x \in Y \wedge Y \in \W \wedge y \not \in
Y\}$. Folglich gilt auch $Z \in \{Y-y ~\vert~ y \in Y \wedge Y \in
\W\}$. Wegen $Z-x=Y$ ist $Y \in  \{Y-x ~\vert~ x \in Y \wedge Y \in \W \wedge y \not \in
Y\}$. Dann gibt es $Z' \in W$ mit $y \in Z'$ und $Y = Z'-y$. Dann ist
$Y+y=Z'-y+y=Z' \in \W$.

\qed
\end{bw}
Die Voraussetzung f\"ur solch ein Vorgehen ist eine Operation
\texttt{Remove} auf QOBDDs, so dass f\"ur alle Knoten $a$ und alle $i
\in \{1,\dots,n\}$ gilt:
\begin{align*}
\set(\texttt{Remove}(a,i))) = \{Y-x_i ~\vert~ Y \in \set(a) \wedge x_i
\in Y\}
\end{align*}

\begin{ml}
\begin{itemize}
\item[(1)] F\"ur alle QOBDDs $\ith(i)$ gilt:
\begin{align*}
\set(\texttt{Neg}(\ith(i)) = 2^X \setminus \set(\ith(i)) = 2^{X-x_i}
\end{align*}
\item[(2)] F\"ur alle QOBDDs $g_1$ und $g_2$ gilt mit $g_1 \sqcap g_2
  = \texttt{Apply}(g_1,g_2,\cdot)$:
\begin{align*}
\set(g_1 \sqcap g_2) = \set(g_1) \cap \set(g_2)
\end{align*}
\end{itemize}
\qed
\end{ml}

\begin{ms}
Es sei \sg einfaches Spiel. Dann sind f\"ur alle $x_i, x_j \in X$ mit
$i \not = j$ die folgenden Aussagen \"aquivalent:
\begin{align*}
\text{ (1) }& x_i \ww x_j\\
\text{ (2) }& \text{Ist } g \text{ das QOBDD zur Darstellung von } (X,\W) \text{
  mit Wurzel } r \text{\, so ist }\\
& \texttt{Remove}(r \sqcap
\texttt{Neg}(\ith(j)),i) \sqcap \texttt{Neg}(\texttt{Remove}(r,j)) \text{
  gleich dem QOBDD zur Darstellung von } \0: \bol
\end{align*}
\end{ms}

\begin{bw}
Sei $r$ Wurzel von $g$. Dann gilt $\set(r) = \W$.
\begin{align*}
x_i \ww x_j \Leftrightarrow&~ \{Y-x_i ~\vert~ x_i \in Y \wedge Y \in \W \wedge x_j \not
\in Y\} \subseteq \{Y-x_j ~\vert~ x_j \in Y \wedge Y \in \W\}\\
\Leftrightarrow&~ \{Y-x_i ~\vert~ x_i \in Y \wedge Y \in \set(r) \wedge
x_j \not \in Y\} \subseteq \{Y-x_j ~\vert~ x_j \in Y \wedge Y \in
\set(r)\}
\end{align*}
\begin{align*}
\{Y-x_i ~\vert~ x_i \in Y \wedge Y \in \set(r) \wedge
x_j \not \in Y\} =&~ \{Y-x_i ~\vert~ Y \in 2^{X-x_j} \wedge Y \in
\set(r) \wedge x_i \in Y\}\\
=&~ \{Y-x_i ~\vert~ Y \in \set(\texttt{Neg}(\ith(j))) \wedge Y \in
\set(r) \wedge x_i \in Y\} \\
=&~ \{Y-x_i ~\vert~ Y \in \set(r \sqcap \texttt{Neg}(\ith(j))) \wedge
x_i \in Y\} \\
=&~ \set(\texttt{Remove}(r \sqcap \texttt{Neg}(\ith(j)),i))
\end{align*}
\begin{align*}
\{Y-x_j ~\vert~ x_j \in Y \wedge Y \in \set(r)\} =&~ \set(\texttt{Remove}(r,j))
\end{align*}

Also gilt nun insgesamt:
\begin{align*}
x_i \ww x_j \Leftrightarrow&~ \set(\underbrace{\texttt{Remove}(r \sqcap
\texttt{Neg}(\ith(j)), i)}_{f_1: \bol}) \subseteq
\set(\underbrace{\texttt{Remove}(r,j)}_{f_2: \bol})\\
\Leftrightarrow&~ f_1 \sqsubseteq f_2\\
\Leftrightarrow&~ f_1 \sqcap \overline{f_2} = \0\\
\Leftrightarrow&~ \texttt{Remove}(r \sqcap
\texttt{Neg}(\ith(j)), i) \sqcap
\texttt{Neg}(\texttt{Remove}(r,j)) \\
&\text{ ist QOBDD zur
  Darstellung der Funktion } \0: \bol
\end{align*}
\qed
\end{bw}

\begin{ex}
$\set(r) = \{\{x_1,x_2,x_3\}, \{x_1,x_3\}, \{x_2,x_3\}\}$\\
$\set(\texttt{Remove}(r,1)) = \{\{x_2,x_3\},\{x_3\}\}$\\
Bildchen...
\end{ex}

\begin{ms}
Die Funktion
\begin{align*}
\texttt{Remove}(a,i) = 
\begin{cases} 
\cons(x_i,\mathds{O}_{i+1},\then(a)) &, \text{ falls } \var(a)=x_i\\
\cons(\var(a),\texttt{Remove}(\then(a),i), \texttt{Remove}(\el(a),i))
  &, \text{ falls } \var(a) > x_i \\
a &, \text{ sonst}
\end{cases}
\end{align*}
erf\"ullt die Gleichung 
\begin{align*}
\set(\texttt{Remove}(a,i)) = \{Y-x_i ~\vert~ Y \in \set(a) \wedge x_i
\in Y\}
\end{align*}
\end{ms}

\begin{algrem}
\texttt{Remove}($a$,$i$)
\begin{minipage}{0.85\textwidth}
\begin{empheq}[box=\fbox]{align*}
\text{if } \var(a) = x_i &\text{ then return } (\cons(x_i, \mathds{O}_{i+1},
\then(a))\\
&\text{ elseif } \exists r \text{ in CT with } (a,i,r) \text{ then
  return } r \text{ with } (a,i,r) \text{ in CT }\\
&\text{ else } r:= \cons(\var(a), \texttt{Remove}(\then(a),i),
\texttt{Remove}(\el(a),i))\\
&\text{\hspace{0.7cm} add } (a,i,r) \text{ to CT }\\
&\text{\hspace{0.7cm} return } r
\end{empheq}
\end{minipage}
\end{algrem}

\begin{mex}
Sei $X = \{x_1, x_2, x_3,x_4\}, \pi = (x_1,x_2,x_3)$\\
$\set(a) = \{\{x_1,x_2,x_3,x_4\}, \{x_1,x_2,x_4\}, \{x_1,x_3,x_4\},
\{x_1,x_4\}, \{x_2,x_3,x_4\}, \{x_2,x_4\}, \{x_3,x_4\}, \{x_4\}\}$\\
$\set(\texttt{Remove}(a,3))= \{\{x_1,x_2,x_4\},\{x_1,x_4\},
\{x_2.x_4\}, \{x_4\}\}$
\end{mex}

\begin{mex}
Siehe Tabelle \ref{tab:wwr}.
\begin{table}
\begin{centering}
\begin{tabular}[h]{|c|c|c|c|c|c|c|}
\hline $\ww$ & CDU/CSU & SPD & FDP & Die Linke & Die Gr\"unen \\
\hline CDU/CSU & X & O & O & O & O \\
\hline SPD & X & X & Y & Y & O \\
\hline FDP & X & X & X & Y & O \\
\hline Die Linke & X & X & X & X & O \\
\hline Die Gr\"unen & X & X & X & X & X \\
\hline
\end{tabular}


\begin{tabular}[h]{|c|c|c|c|c|c|c|}
\hline $\approx_I$ & CDU/CSU & SPD & FDP & Die Linke & Die Gr\"unen \\
\hline CDU/CSU & X & O & O & O & O \\
\hline SPD & O & X & X & X & O \\
\hline FDP & O & X & X & X & O \\
\hline Die Linke & O & X & X & X & O \\
\hline Die Gr\"unen & O & O & O & O & X \\
\hline
\end{tabular}

\begin{tabular}[h]{|c|c|c|c|c|c|c|}
\hline $\prec_I$ & CDU/CSU & SPD & FDP & Die Linke & Die Gr\"unen \\
\hline CDU/CSU & O & O & O & O & O \\
\hline SPD & X & O & O & O & O \\
\hline FDP & X & O & O & O & O \\
\hline Die Linke & X & O & O & O & O \\
\hline Die Gr\"unen & X & X & X & X & O \\
\hline
\end{tabular}
\caption{Alle Eintr\"age bis auf die Ys ergeben sich aus der Quote
  ($w(x) < w(y) \Rightarrow x \ww y$)}
\end{centering}
\label{tab:wwr}
\end{table}
\end{mex}

\subsection{Eigenschaften von einfachen Spielen}
Bisher wurde nur zwischen GMS und VGMS unterschieden.

\begin{mdef}
Ein einfaches Spiel \sg hei\ss t 
\begin{itemize}
\item[(1)] \underline{echt}, falls f\"ur alle $Y \in \W$ gilt: $X
  \setminus Y \not \in \W$
\item[(2)] \underline{fest}, falls f\"ur alle $Y \in 2^X \setminus W$ gilt: $X
  \setminus Y \in W$
\item[(3)] \underline{entscheidend}, falls es echt \emph{und}
  fest ist
\item[(4)] \underline{dual-gleichwertig}, falls es echt \emph{oder}
  fest ist
\end{itemize}
\qed
\end{mdef}

\begin{ms}
F\"ur alle einfachen Spiele \sg gilt:
\begin{align*}
(X, \W) \text{ entscheidend } \Rightarrow \vert \W \vert =
2^{\vert X \vert -1}
\end{align*}
\end{ms}

\begin{bw}
Definiere $\Phi: 2^X \rightarrow 2^X, ~\Phi(Y) = X \setminus Y$. Dann
gilt $\Phi^2 = id$, also $\Phi$ bijektiv.
\begin{align*}
(X,\W) \text{ echt } \Leftrightarrow&~ \Phi(\W) \subseteq 2^X \setminus \W\\
\Rightarrow&~ \vert \W \vert = \vert \Phi(\W) \vert \leq \vert 2^X
\setminus \W \vert\\
(X,\W) \text{ fest } \Leftrightarrow&~ \Phi(2^X \setminus \W)
\subseteq \W\\
\Rightarrow&~ \vert 2^X \setminus \W \vert = \vert \Phi(2^X \setminus
\W) \vert
\leq \vert W \vert
\end{align*}
\qed
\end{bw}

\begin{ms}
Ist \sg ein GMS, so gilt:
\begin{itemize}
\item[(1)] \sg ist echt oder fest
\item[(2)] $\vert \W \vert = 2^{\vert X \vert -1} \Rightarrow (X,\W)$
  entscheidend
\end{itemize}
\end{ms}

\begin{bw}
(1) Es sei $(Q,w)$ gewichtete Darstellung und \sg weder echt noch
fest.
\begin{align*}
(X,\W) \text{ nicht echt } \Rightarrow \exists Y \in 2^X: Y \in \W
~\wedge~ X \setminus Y \in \W
\end{align*}
Dann gilt f\"ur so ein $Y$:
\begin{align*}
\sum_{x \in Y} w(x) \geq Q \text{ und } \sum_{x \in X \setminus Y}
w(x) \geq X \text{ also auch } \sum_{x \in X} w(x) = \sum_{x \in
  Y} w(x) + \sum_{x \in X \setminus Y} \geq 2 \cdot Q
\end{align*}
\begin{align*}
(X,\W) \text{ nicht fest } \Rightarrow \exists Z \in 2^X: Z \in 2^X
\setminus \W ~\wedge~ X \setminus Z \in 2^X \setminus \W
\end{align*}
Dann gilt f\"ur so ein $Z$:
\begin{align*}
\sum_{x \in Z} w(x) < Q \text{ und } \sum_{x \in X \setminus Z}
w(x) < X \text{ also insgesamt } \sum_{x \in X} w(x) = \sum_{x \in
  Z} w(x) + \sum_{x \in X \setminus Z} < 2 \cdot Q
\end{align*}
Widerspruch!\\

(2)
Sei $\vert \W \vert = 2^{\vert X \vert -1}$, aber $(X,\W)$ nicht
entscheidend.\\
\emph{Fall 1:} Es sei \sg echt, aber nicht fest.
\begin{align*}
(X,W) \text{ echt } \Leftrightarrow&~ \Phi(\W) \subseteq 2^X \setminus
\W \\
\Rightarrow&~ \vert \W \vert \leq \vert 2^x \setminus \W \vert
\end{align*}
Da $\vert \W \vert = 2^{\vert X \vert -1}$ gilt und $(\W, 2^X \setminus \W)$
eine Partition bilden, gilt:
\begin{align*}
\vert W \vert = \vert 2^X \setminus \W \vert = \vert \Phi(\W) \vert
\end{align*}
Also gilt $\Phi(\W) = 2^X \setminus \W$. Und damit auch $\W =
\Phi(\Phi(\W)) = \Phi(2^X \setminus \W)$. Also insgesamt $\Phi(2^X
\setminus \W) \subseteq \W$, d.h. \sg ist fest.\\
Widerspruch!\\

\emph{Fall 2:} Analog.\\
\qed
\end{bw}

\begin{algdec}
Der folgende Algorithmus testet, ob ein GMS \sg entscheidend ist.\\
\texttt{Decisive}($Q$,$w$)
\begin{minipage}{0.8\textwidth}
\begin{empheq}[box=\fbox]{align*}
  \text{(1)}&\text{ Berechne das QOBDD zur Darstellung des Spiels}\\
    &\text{ mit Hilfe von \hyperref[sat:fiq]{Satz \ref{sat:fiq}}}\\
  \text{(2)}&\text{ Z\"ahle in diesem QOBDD die Anzahl } m \text{ der
    erf\"ullenden Belegungen}\\
  &\text{ nach \hyperref[sat:325]{Satz
      \ref{sat:anz}}}\\
  \text{(3)}&\text{ return } m=2^{\vert X \vert -1}
\end{empheq}
\end{minipage}
\newline

\qed
\end{algdec}

\begin{mdef}
Sei \sg einfaches Spiel.
\begin{itemize}
\item[(1)] Eine Koalition $Y \in 2^X$ hei\ss t \emph{blockierend},
  falls $X \setminus Y \not \in \W$.
\item[(2)] Es hei\ss t $\W^d = \{ Y \in 2^X ~|~ Y \text{ blockierend}\}$
  \emph{Dual} von $\W$.
\item[(3)] $(X, \W^d)$ ist das zu \sg \emph{duale einfache} Spiel.
\end{itemize}
\qed
\end{mdef}
Beachte dabei, dass $Y \in \W^d$ und $Y \subseteq Z \Rightarrow Z \in
\W^d$ gilt.

\begin{mex}
(1) (fest und echt) \\
$X = \{x_1,x_2,x_3\}, ~\W =
\{\{x_1,x_2,x_3\},\{x_1,x_2\},\{x_1,x_3\},\{x_2,x_3\}\}$\footnote{Bei $W =
  W^d$ spricht man von einem Selbstdual.}
\begin{table}[h]
\begin{tabular}[h]{|c||c|c|c|c|c|c|c|c|}
\hline $x_1$ & 0 & 0 & 0 & 0 & 1 & 1 & 1 & 1\\
\hline $x_2$ & 0 & 0 & 1 & 1 & 0 & 0 & 1 & 1\\
\hline $x_3$ & 0 & 1 & 0 & 1 & 0 & 1 & 0 & 1\\
\hline $\W$ & 0 & 0 & 0 &1 & 0 & 1 & 1 & 1\\
\hline $\W^d$ & 0 & 0 & 0 & 1 & 0 & 1 & 1 & 1\\
\hline
\end{tabular}
\end{table}

(2) (nicht fest, aber echt)\\
$X = \{x_1, x_2,x_3\}, ~\W = \{\{x_1, x_2, x_3\},\{x_2,x_3\}\}$
\begin{table}[h]
\begin{tabular}[h]{|c||c|c|c|c|c|c|c|c|}
\hline $x_1$ & 0 & 0 & 0 & 0 & 1 & 1 & 1 & 1\\
\hline $x_2$ & 0 & 0 & 1 & 1 & 0 & 0 & 1 & 1\\
\hline $x_3$ & 0 & 1 & 0 & 1 & 0 & 1 & 0 & 1\\
\hline $\W$ & 0 & 0 & 0 &1 & 0 & 0 & 0 & 1\\
\hline $\W^d$ & 0 & 1 & 1 & 1 & 0 & 1 & 1 & 1\\
\hline
\end{tabular}
\end{table}
\end{mex}

\begin{ms}
Es sei \sg einfaches Spiel. Dann gelten die \"Aquivalenzen:
\begin{itemize}
\item[(1)] $(X,\W)$ echt $\Leftrightarrow \W \subseteq \W^d$
\item[(2)] $(X,\W)$ fest $\Leftrightarrow \W^d \subseteq \W$
\end{itemize}
\end{ms}

\begin{bw}
(1) \glqq $\Rightarrow$\grqq~ Sei $Y \in 2^X$, dann gilt:
\begin{align*}
Y \in \W \Rightarrow&~ X \setminus Y \in 2^X \setminus \W & \text{
  wegen } (X,\W)
\text{ echt}\\
\Leftrightarrow&~ Y \in \W^d & \text{ Def. von } \W^d
\end{align*}
Also $\W \subseteq W^d$.\\
\glqq $\Leftarrow$\grqq~ Seien $\W \subseteq W^d$ und $Y \in 2^X$.
\begin{align*}
Y \in W \Rightarrow&~ Y \in W^d & \text{ wegen } \W \subseteq \W^d\\
\Leftrightarrow&~ X \setminus Y \in 2^x \setminus \W & \text{ Def. von
 echt}
\end{align*}
Also ist \sg echt.\\

(2) Analog.\\
% \glqq $\Rightarrow$\grqq~ Sei $Y \in 2^X $, dann gilt:
% \begin{align*}
% Y \in 2^X \setminus \W \Rightarrow&~ X \setminus Y  \in \W\\
% \Leftrightarrow&~ X \setminus Y \not \in  2^X \setminus \W
% \end{align*}
% Also $W^d \subseteq \W$.\\
% \glqq $\Leftarrow$\grqq~ Seien $W^d \subseteq \W$ und $Y \in 2^X$.
% \begin{align*}
% Y \in \W^d \Rightarrow&~ Y \in \W \\
% \Leftrightarrow&~ Y \not \in 2^X \setminus \W
% \end{align*}

\qed
\end{bw}
F\"ur $a$ als QOBDD-Knoten mit $\var(a)=x_i$ soll im Folgenden gelten:
\begin{align*}
\set(\texttt{Compl}(a)) = \{ Y \in 2^{\{x_i,\dots,x_n\}}
~|~ \{x_i, \dots, x_n\} \setminus Y \in \set(a)\}
\end{align*}

\begin{ms}
Es sei \sg einfaches Spiel, dargestellt durch QOBDD mit Wurzel
$r$. Dann gilt
\begin{align*}
(X,\W) \text{ echt } \Leftrightarrow \text{ Das QOBDD }
\texttt{Compl}(r) \sqcap r \text{ ist gleich dem QOBDD zur Darstellung
  von } \0
\end{align*}
\end{ms}

\begin{bw}
\begin{align*}
(X,\W) \text{ echt } \Leftrightarrow&~ \W \subseteq \W^d & \text{
  Def. von echt}\\
\Leftrightarrow&~ W \subseteq \{Y \in 2^X ~|~ X \setminus Y \not \in
\W\} & \text{ Def. von } \W^d\\
\Leftrightarrow&~ \set(r) \subseteq \{Y \in 2^X ~|~ X \setminus Y \not
\in \set(r)\} & \W = \set(r)\\
\Leftrightarrow&~ \set(r) \subseteq \{Y \in 2^x ~|~ X \setminus Y \in
\set(\texttt{Neg}(r))\} & \text{ Eig. von \texttt{Neg} }\\
\Leftrightarrow&~ \set\underbrace{(r)}_{f_1} \subseteq
\set(\underbrace{\texttt{Compl}(\texttt{Neg}(r))}_{f_2}) &
\text{(*)}\\
\Leftrightarrow&~ f_1 \sqsubseteq f_2 \\
\Leftrightarrow&~ \overline{f_2} \sqcap f_1 = \0\\
\Leftrightarrow&~ \texttt{Neg}(\texttt{Compl}(\texttt{Neg}(r))) \sqcap
r \text{ ist QOBDD f\"ur } \0\\
\Leftrightarrow&~ \texttt{Compl}(r) \sqcap r \text{ ist QOBDD f\"ur } \0
\end{align*}
\qed
\end{bw}

\begin{ms}
Es sei \sg einfaches Spiel, dargestellt durch QOBDD mit Wurzel
$r$. Dann gilt:
\begin{align*}
(X,\W) \text{ fest } \Leftrightarrow \text{Das QOBDD }
\texttt{Neg}(\texttt{Compl}(r) \sqcap \texttt{Neg}(r)) \text{ ist gleich
  dem QOBDD zur Darstellung von } \0
\end{align*}
\end{ms}

\begin{bw}
\begin{align*}
(X,\W) \text{ fest } \Leftrightarrow&~ \W^d \subseteq \W \\
\Leftrightarrow&~ \{Y \in 2^X ~|~ X \setminus Y \not \in \W \}
\subseteq \W\\
\&~ \{Y \in 2^X ~|~ X \setminus Y \not \in \set(r) \}
\subseteq \set(r)\\
\Leftrightarrow&~ \{Y \in 2^X ~|~ X \setminus Y \in
\set(\texttt{Neg} (r)\} \subseteq \set(r)\\
\Leftrightarrow&~
\set(\underbrace{\texttt{Compl}(\texttt{Neg}(r)}_{f_1})
\subseteq \set\underbrace{(r)}_{f_2}\\
\Leftrightarrow&~ f_1 \sqsubseteq f_2\\
\Leftrightarrow&~ \overline{f_2} \sqcap f_1 = \0\\
\Leftrightarrow&~ \texttt{Neg}(r) \sqcap
\texttt{Compl}(\texttt{Neg}(r)) \text{ ist das QOBDD f\"ur } \0
\end{align*}
\qed
\end{bw}

Sei $a$ wieder QOBDD-Knoten mit $\var(a) = x_i$. Dann wissen wir
schon, dass gilt:
\begin{align*}
\set(a) = \{eins(w) ~|~ w \text{Weg von } a \text{ nach } \mathds{I}\}
\end{align*}
Daraus k\"onnen wir nun observieren, dass folgende Gleichheit f\"ur
das Komplement gelten muss:
\begin{align*}
&\set(\texttt{Compl}(a)) =\{null(w) ~|~ w \text{ Weg von } a \text{
  nach } \mathds{I}\}\\
&\text{mit } null(p_1,\dots,p_m) =
\{\var(\alpha(p_i)) ~|~ p_i \text{ ist 0-Pfeil } \wedge 1 \leq i \leq m\}
\end{align*}

\begin{ms}
Sei $a$ QOBDD-Knoten. Die Funktion \texttt{Compl} mit
\begin{align*}
\texttt{Compl}(a) = \begin{cases} \mathds{I} &\text{, falls }
  a=\mathds{I}\\
\mathds{O} &\text{, falls } a = \mathds{O}\\
\cons(\var(a),\texttt{Compl}(\el(a)), \texttt{Compl}(\then(a)))
&\text{, sonst}
\end{cases}
\end{align*}
erf\"ullt die Eigenschaft $\set(\texttt{Compl}(a)) = \{ Y \in
2^{\{x_i,\dots,x_n\}} ~|~ \{x_i,\dots,x_n\} \setminus Y \in \set(a)\}$.
\qed
\end{ms}

\begin{mex}
Sei einfaches Spiel \sg gegeben mit $X = \{x_1,x_2,x_3,x_4\}$ und $\W$
wie folgt:
\begin{table}[h]
\begin{tabular}[h]{|c||c|c|c|c|c|c|c|c|c|c|c|c|c|c|c|c|}
\hline $x_1$ &   0&0&0&0&0&0&0&0&1&1&1&1&1&1&1&1 \\
\hline $x_2$ &   0&0&0&0&1&1&1&1&0&0&0&0&1&1&1&1 \\
\hline $x_3$ &   0&0&1&1&0&0&1&1&0&0&1&1&0&0&1&1 \\
\hline $x_4$ &   0&1&0&1&0&1&0&1&0&1&0&1&0&1&0&1 \\
\hline \hline $\W$ &    0&0&0&0&0&0&1&1&0&1&0&1&1&1&1&1 \\
\hline $\W^d$ &1&1&1&1&1&0&1&0&1&1&0&0&0&0&0&0 \\
\hline
\end{tabular}
\end{table}
\newline
Dieses Spiel ist nicht fest und nicht echt.\\
Sei $Z := \{x_2,x_4\}$, dann gilt $Z \not \in \W$ und $X \setminus Z
\not \in \W$, also ist \sg nicht fest.\\
Sei dann $Y := \{x_2,x_3\}$, dann gilt $Y \in \W$ und $X \setminus Y
\in \W$, also \sg auch nicht echt.
\end{mex} 
\end{document}

